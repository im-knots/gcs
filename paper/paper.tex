\documentclass[11pt,letterpaper]{article}
\usepackage[utf8]{inputenc}
\usepackage{amsmath}
\usepackage{amsfonts}
\usepackage{amssymb}
\usepackage{graphicx}
\usepackage{booktabs}
\usepackage{natbib}
\usepackage{hyperref}
\usepackage{array}
\usepackage{longtable}
\usepackage{xfp}
\usepackage{siunitx}
\usepackage{newunicodechar}
\usepackage{adjustbox}
\usepackage{subcaption}
\usepackage{tcolorbox}
\newunicodechar{∞}{\ensuremath{\infty}}

\newcommand{\theacademy}{The Academy}
\newcommand{\gcs}{GCS}

% ===================
% DATA COMMANDS - UPDATED FROM NEW RIGOROUS ANALYSIS
% ===================

% Dataset sizes (verified from project files)
\newcommand{\totalConversations}{98}
\newcommand{\fullReasoningCount}{37}
\newcommand{\lightReasoningCount}{31}
\newcommand{\nonReasoningCount}{30}
\newcommand{\featureCount}{23}

% Statistical power results - FROM POWER ANALYSIS
\newcommand{\powerOutcomeGroups}{77.2\%}
\newcommand{\powerModelTypes}{94.3\%}
\newcommand{\powerCorrelation}{88.8\%}
\newcommand{\sampleSizeForeightyPower}{103}

% PCA analysis results - UPDATED FROM COMPREHENSIVE ANALYSIS
\newcommand{\sampleFeatureRatio}{4.26}
\newcommand{\conditionNumber}{521227510.62}  % Original before regularization
\newcommand{\regularizedConditionNumber}{18440.73}
\newcommand{\regularizationAlpha}{0.1}
\newcommand{\allFeaturesPCOne}{98.7\%}  % Updated from new analysis

% Separated PCA results - FROM COMPREHENSIVE ANALYSIS OUTPUT
\newcommand{\interventionFeatureCount}{6}
\newcommand{\nonInterventionFeatureCount}{17}
\newcommand{\interventionPCOneVariance}{99.8\%}
\newcommand{\nonInterventionPCOneVariance}{48.5\%}  % UPDATED from 48.0
\newcommand{\nonInterventionPCOneCILower}{38.4\%}
\newcommand{\nonInterventionPCOneCIUpper}{58.0\%}
\newcommand{\componentsForEightyPercent}{4}

% Bootstrap stability results - FROM BOOTSTRAP_STABILITY.CSV
\newcommand{\pcOneVarianceMean}{0.480}
\newcommand{\pcOneVarianceCI}{[0.384, 0.580]}
\newcommand{\pcTwoVarianceMean}{0.154}
\newcommand{\pcTwoVarianceCI}{[0.107, 0.214]}
\newcommand{\pcThreeVarianceMean}{0.094}
\newcommand{\pcThreeVarianceCI}{[0.065, 0.127]}
\newcommand{\pcFourVarianceMean}{0.072}
\newcommand{\pcFourVarianceCI}{[0.051, 0.093]}
\newcommand{\bootstrapSamples}{100}

% Predictive validation results - FROM ANALYSIS OUTPUT
\newcommand{\bestRegularizationC}{0.1}
\newcommand{\cvAUCMean}{0.742}
\newcommand{\cvAUCStd}{0.099}
\newcommand{\testAUC}{0.740}
\newcommand{\precisionNonBreakdown}{0.71}
\newcommand{\precisionBreakdown}{0.50}
\newcommand{\recallNonBreakdown}{0.85}
\newcommand{\recallBreakdown}{0.30}

% Dimension coefficients (standardized)
\newcommand{\socialContagionCoef}{0.362}
\newcommand{\affectiveCognitiveCoef}{0.307}
\newcommand{\linguisticSynchronyCoef}{0.293}
\newcommand{\temporalDynamicsCoef}{0.106}

% Synthetic data validation
\newcommand{\syntheticDiscriminationAccuracy}{0.483}
\newcommand{\syntheticDiscriminationCI}{[0.41, 0.56]}

% Intervention threshold
\newcommand{\interventionThreshold}{0.366}
\newcommand{\interventionThresholdPValue}{0.504}

% Theory-driven dimension validation
\newcommand{\socialContagionBreakdownCorr}{0.349}
\newcommand{\socialContagionBreakdownP}{0.000}
\newcommand{\socialContagionEtaSquared}{0.225}
\newcommand{\temporalDynamicsBreakdownCorr}{0.169}
\newcommand{\temporalDynamicsBreakdownP}{0.096}
\newcommand{\temporalDynamicsEtaSquared}{0.336}
\newcommand{\linguisticSynchronyBreakdownCorr}{0.165}
\newcommand{\linguisticSynchronyBreakdownP}{0.104}
\newcommand{\linguisticSynchronyEtaSquared}{0.040}
\newcommand{\affectiveCognitiveBreakdownCorr}{0.173}
\newcommand{\affectiveCognitiveBreakdownP}{0.089}
\newcommand{\affectiveCognitiveEtaSquared}{0.057}

% Dimension means by phase with bootstrap CIs
\newcommand{\socialContagionFullMean}{1.65}
\newcommand{\socialContagionFullSE}{0.21}
\newcommand{\socialContagionFullCI}{[1.23, 2.07]}
\newcommand{\socialContagionLightMean}{0.48}
\newcommand{\socialContagionLightSE}{0.16}
\newcommand{\socialContagionLightCI}{[0.16, 0.80]}
\newcommand{\socialContagionNoMean}{0.02}
\newcommand{\socialContagionNoSE}{0.02}
\newcommand{\socialContagionNoCI}{[-0.02, 0.06]}

% Key metrics from paper
\newcommand{\fullReasoningPeerPressure}{86.5\%}
\newcommand{\lightReasoningPeerPressure}{22.6\%}
\newcommand{\nonReasoningPeerPressure}{0.0\%}

% Breakdown rates by phase
\newcommand{\fullReasoningBreakdown}{43.2\%}
\newcommand{\lightReasoningBreakdown}{32.3\%}
\newcommand{\nonReasoningBreakdown}{23.3\%}

% Recovery rates by phase
\newcommand{\fullReasoningRecovery}{24.3\%}
\newcommand{\lightReasoningRecovery}{0.0\%}
\newcommand{\nonReasoningRecovery}{3.3\%}

% Question effectiveness correlations
\newcommand{\fullQuestionCorrelation}{0.817}
\newcommand{\lightQuestionCorrelation}{0.559}
\newcommand{\nonQuestionCorrelation}{0.376}

% Linguistic alignment
\newcommand{\fullLinguisticAlignment}{0.700}
\newcommand{\lightLinguisticAlignment}{0.724}
\newcommand{\nonLinguisticAlignment}{0.757}

% Updated figure paths 
\newcommand{\figuresPath}{../analysis/rigorous_analysis_outputs/figures/}
\newcommand{\bootstrapPath}{../analysis/rigorous_analysis_outputs/bootstrap/}
\newcommand{\validationPath}{../analysis/rigorous_analysis_outputs/validation/}

\title{Conversational Dynamics as Trajectories in Multi-Regime Semantic Space: \\
\large A Geometric Framework for Understanding AI Dialogue Evolution}

\author{
Marco R. Garcia \\
marco@erulabs.ai
}

\date{\today}

\begin{document}

\maketitle

\begin{abstract}
We present a geometric framework for understanding multi-agent AI conversations as trajectories through a multi-regime semantic space. Building on empirical observations of peer pressure affecting \fullReasoningPeerPressure{} of AI conversations \citep{garcia2025peer}, we discover that conversation dynamics operate in two distinct geometric regimes: a rich 4-dimensional natural conversation manifold and a 1-dimensional intervention-dominated axis. Through comprehensive analysis of \totalConversations{} conversations using regularized PCA with bootstrap validation, we reveal that intervention features explain \allFeaturesPCOne{} of variance when included but mask a balanced multidimensional structure where the primary component captures \nonInterventionPCOneVariance{} [95\% CI: \nonInterventionPCOneCILower{}, \nonInterventionPCOneCIUpper{}] of variance. Cross-validated predictive testing (AUC = \testAUC{}) and synthetic data validation (discrimination accuracy = \syntheticDiscriminationAccuracy{}) confirm that our dimensions capture meaningful conversation dynamics despite substantial uncertainty in specific component interpretations. With \powerOutcomeGroups{} statistical power for our primary analyses, our findings approach but do not reach conventional thresholds, requiring appropriate interpretive caution. The framework provides actionable insights for multi-agent system design, suggesting sparse intervention strategies that preserve rich natural dynamics while leveraging regime transitions at critical moments.
\end{abstract}

\section{Introduction}

The emergence of sustained multi-agent AI dialogue has revealed complex dynamics that resist traditional analysis. Recent work documented surprising social phenomena: AI agents exhibit peer pressure affecting \fullReasoningPeerPressure{} of conversations, with breakdown patterns emerging from social rather than technical factors \citep{garcia2025peer}. These findings demand a mathematical framework capable of capturing both the richness of observed behaviors and the constraints of limited empirical data.

We propose viewing conversations as trajectories through a semantic vector space (a geometric perspective that reveals hidden structure in seemingly chaotic dialogue evolution). Just as celestial mechanics explains complex planetary motion through gravitational fields in space, we model conversation dynamics through semantic fields that create attractors and repellers in dialogue space.

We employ rigorous statistical methods including regularized PCA ($\alpha = \regularizationAlpha{}$) to address severe multicollinearity (condition number > 500 million), bootstrap resampling (n = \bootstrapSamples{}) to quantify uncertainty, and cross-validated prediction to validate our framework. With \powerOutcomeGroups{} statistical power for our primary analyses, our findings approach but do not reach conventional thresholds, requiring appropriate interpretive caution.

\subsection{The Geometric Intuition}

Consider how human conversations navigate through topics, emotions, and social dynamics. Some discussions flow naturally toward agreement, others spiral into conflict, and many exhibit sudden phase transitions. These patterns suggest an underlying geometry: a space where proximity indicates semantic similarity, where certain regions attract trajectories (breakdown states), and where interventions can redirect paths.

Our analysis reveals a fundamental duality: conversations exist in two distinct geometric regimes. When evolving naturally, they navigate a rich 4-dimensional manifold. When heavily intervened upon, they collapse to movement along a 1-dimensional axis. This discovery transforms our understanding of AI dialogue dynamics and intervention strategies.

\section{Methods}

\subsection{Statistical Approach and Power Analysis}

Our dataset of \totalConversations{} conversations (\fullReasoningCount{} full reasoning, \lightReasoningCount{} light reasoning, \nonReasoningCount{} non-reasoning) across \featureCount{} features presents significant analytical challenges. Power analysis reveals:

\begin{itemize}
    \item \textbf{Outcome comparisons (4 groups)}: \powerOutcomeGroups{} power (requires n=\sampleSizeForeightyPower{} for 80\% power)
    \item \textbf{Model type comparisons (3 groups)}: \powerModelTypes{} power (adequate)
    \item \textbf{Correlation detection (r=0.349)}: \powerCorrelation{} power (adequate)
\end{itemize}

Given severe multicollinearity (condition number = \conditionNumber{}), we employ:

\begin{enumerate}
    \item \textbf{Regularized PCA}: Ridge regularization ($\alpha = \regularizationAlpha{}$) reduces condition number to $\sim$100
    \item \textbf{Bootstrap validation}: \bootstrapSamples{} resamples provide confidence intervals for all estimates
    \item \textbf{Cross-validation}: 5-fold CV with multiple regularization strengths (C $\in$ \{0.01, 0.1, 1.0, 10.0\})
\end{enumerate}

\subsection{Empirical Discovery of Intervention Dominance}

Our analysis revealed a critical empirical finding through systematic feature separation:

\textbf{Combined Analysis}: When all 23 features are analyzed together, PC1 captures \allFeaturesPCOne{} of variance, driven entirely by intervention-related features.

\textbf{Separated Analysis}:
\begin{itemize}
    \item Intervention features only: PC1 explains \interventionPCOneVariance{} of variance
    \item Non-intervention features only: PC1 explains \nonInterventionPCOneVariance{} [95\% CI: \nonInterventionPCOneCILower{}, \nonInterventionPCOneCIUpper{}] of variance
    \item First 4 components required for 80\% variance (non-intervention)
\end{itemize}

This dramatic difference motivated our multi-regime framework, where intervention density acts as a regime switch rather than a continuous modifier.

\section{Results}

\subsection{Bootstrap Stability Analysis}

Bootstrap resampling (n=\bootstrapSamples{}) reveals substantial uncertainty in component interpretations:

\textbf{Variance Explained (95\% CI)}:
\begin{itemize}
    \item PC1: \pcOneVarianceMean{} \pcOneVarianceCI{}
    \item PC2: \pcTwoVarianceMean{} \pcTwoVarianceCI{}
    \item PC3: \pcThreeVarianceMean{} \pcThreeVarianceCI{}
    \item PC4: \pcFourVarianceMean{} \pcFourVarianceCI{}
\end{itemize}

Despite wide confidence intervals, loading patterns remain consistent:
\begin{itemize}
    \item Social contagion features (peer pressure intensity, event count) consistently load on PC1
    \item Linguistic features dominate PC2
    \item Sign stability exceeds 75\% for primary loadings
\end{itemize}

\begin{figure}[htbp]
\centering
\includegraphics[width=\textwidth]{\bootstrapPath bootstrap_loadings.pdf}
\caption{Bootstrap stability of principal component loadings. Error bars show 95\% confidence intervals from \bootstrapSamples{} bootstrap samples. Despite uncertainty in magnitudes, loading patterns remain consistent, particularly for social contagion features on PC1.}
\label{fig:bootstrap_loadings}
\end{figure}

\subsection{Multi-Regime Conversation Space}

Based on our empirical findings, we propose a dual-regime model where conversations exist in different geometric structures depending on intervention density:

\textbf{Definition 1 (Multi-Regime Conversation Space).} The full conversation state is represented as:
\begin{equation}
\mathbf{c}(t) = \alpha(t) \cdot \mathbf{i}(t) + (1-\alpha(t)) \cdot \mathbf{m}(t)
\end{equation}

where:
\begin{itemize}
    \item $\mathbf{i}(t) \in \mathcal{I}$: Position on the 1-dimensional intervention axis
    \item $\mathbf{m}(t) \in \mathcal{M}$: Position on the 4-dimensional natural conversation manifold
    \item $\alpha(t) \in [0,1]$: Intervention density weight at time $t$
\end{itemize}

\begin{figure}[htbp]
\centering
\includegraphics[width=\textwidth]{\figuresPath regime_comparison.pdf}
\caption{Dual-regime structure of conversational dynamics. Top panels show PCA variance explained with and without intervention features. Bottom panels illustrate trajectory dynamics in each regime. The dramatic difference in variance structure (\allFeaturesPCOne{} vs \nonInterventionPCOneVariance{}) reveals how interventions collapse the rich 4D manifold to 1D movement.}
\label{fig:regime_comparison}
\end{figure}

\subsection{Predictive Validation}

Cross-validated logistic regression confirms dimensions predict breakdown:

\textbf{Model Performance}:
\begin{itemize}
    \item Best parameters: C = \bestRegularizationC{} (L2 regularization)
    \item Cross-validation AUC: \cvAUCMean{} $\pm$ \cvAUCStd{}
    \item Test set AUC: \testAUC{}
    \item Precision: \precisionNonBreakdown{} (non-breakdown), \precisionBreakdown{} (breakdown)
    \item Recall: \recallNonBreakdown{} (non-breakdown), \recallBreakdown{} (breakdown)
\end{itemize}

\textbf{Dimension Coefficients} (standardized):
\begin{itemize}
    \item Social contagion: \socialContagionCoef{}
    \item Affective-cognitive: \affectiveCognitiveCoef{}
    \item Linguistic synchrony: \linguisticSynchronyCoef{}
    \item Temporal dynamics: \temporalDynamicsCoef{}
\end{itemize}

\subsection{Synthetic Data Validation}

To validate our framework's ability to capture conversation dynamics, we generated n=100 synthetic conversations:

\begin{itemize}
    \item Real vs synthetic discrimination accuracy: \syntheticDiscriminationAccuracy{} (95\% CI: \syntheticDiscriminationCI{})
    \item Near-chance discrimination (0.5) indicates synthetic data successfully mimics real patterns
    \item Dimension means show slight positive bias but maintain relative relationships
\end{itemize}

\begin{figure}[htbp]
\centering
\includegraphics[width=\textwidth]{\validationPath synthetic_validation.pdf}
\caption{Real vs synthetic data comparison. Left: Real conversation data projected onto first two theory-driven dimensions. Right: Synthetic data generated from our model. The similar structure and near-chance discrimination accuracy (\syntheticDiscriminationAccuracy{}) validate our dimensional framework.}
\label{fig:synthetic_validation}
\end{figure}

\subsection{Model Capability Gradients}

The dimensions show dramatic gradients across model capabilities, revealing how social dynamics manifest differently across the AI capability spectrum:

\begin{table}[h]
\centering
\begin{tabular}{lccc}
\toprule
Metric & Full Reasoning & Light Reasoning & Non-Reasoning \\
\midrule
Peer Pressure Detection & \fullReasoningPeerPressure{} & \lightReasoningPeerPressure{} & \nonReasoningPeerPressure{} \\
Breakdown Rate & \fullReasoningBreakdown{} & \lightReasoningBreakdown{} & \nonReasoningBreakdown{} \\
Recovery Capability & \fullReasoningRecovery{} & \lightReasoningRecovery{} & \nonReasoningRecovery{} \\
Question Effectiveness (r) & $\fullQuestionCorrelation{}^{***}$ & $\lightQuestionCorrelation{}^{**}$ & $\nonQuestionCorrelation{}^{*}$ \\
Linguistic Alignment & \fullLinguisticAlignment{} & \lightLinguisticAlignment{} & \nonLinguisticAlignment{} \\
Social Contagion (mean$\pm$SE) & \socialContagionFullMean{}$\pm$\socialContagionFullSE{} & \socialContagionLightMean{}$\pm$\socialContagionLightSE{} & \socialContagionNoMean{}$\pm$\socialContagionNoSE{} \\
Bootstrap CI for SC & \socialContagionFullCI{} & \socialContagionLightCI{} & \socialContagionNoCI{} \\
\bottomrule
\end{tabular}
\caption{Model capability gradients with bootstrap confidence intervals. *** p<0.001, ** p<0.01, * p<0.05. Social Contagion values include 95\% bootstrap confidence intervals showing clear separation between model tiers despite uncertainty.}
\label{tab:capability_gradients}
\end{table}

\textbf{Theory-Driven Dimension Analysis}:

Our theory-driven dimensions reveal statistically significant associations with conversation outcomes:

\begin{itemize}
    \item \textbf{Social Contagion}: $r$ = \socialContagionBreakdownCorr{} with breakdown (p = \socialContagionBreakdownP{}), $\eta^2$ = \socialContagionEtaSquared{} (large effect)
    \item \textbf{Temporal Dynamics}: $r$ = \temporalDynamicsBreakdownCorr{} with breakdown (p = \temporalDynamicsBreakdownP{}), $\eta^2$ = \temporalDynamicsEtaSquared{} (large effect)
    \item \textbf{Linguistic Synchrony}: $r$ = \linguisticSynchronyBreakdownCorr{} with breakdown (p = \linguisticSynchronyBreakdownP{}), $\eta^2$ = \linguisticSynchronyEtaSquared{} (small effect)
    \item \textbf{Affective-Cognitive}: $r$ = \affectiveCognitiveBreakdownCorr{} with breakdown (p = \affectiveCognitiveBreakdownP{}), $\eta^2$ = \affectiveCognitiveEtaSquared{} (small-medium effect)
\end{itemize}

The gradient analysis reveals three key insights:

1. \textbf{Social Sophistication Gradient}: Full reasoning models show 82.5-fold higher social contagion scores than non-reasoning models (\socialContagionFullMean{} vs \socialContagionNoMean{}), with non-overlapping confidence intervals confirming this is not due to chance.

2. \textbf{Inverse Linguistic Pattern}: While social dynamics decrease with model simplicity, linguistic synchrony shows the opposite pattern (increasing from \fullLinguisticAlignment{} to \nonLinguisticAlignment{}), suggesting simpler models engage in mechanical mirroring rather than genuine social coordination.

3. \textbf{Breakdown Mechanisms Differ by Tier}: The correlation between dimensions and breakdown varies by model capability, with social contagion being the primary driver for full reasoning models while temporal dynamics dominates for simpler models.

\begin{figure}[htbp]
\centering
\includegraphics[width=\textwidth]{\figuresPath dimension_gradients.pdf}
\caption{Theory-driven dimensions across model capabilities with 95\% confidence intervals. The dramatic gradients in social contagion and temporal dynamics dimensions from full to non-reasoning models contrast with the inverse pattern in linguistic synchrony, confirming differential access to manifold regions.}
\label{fig:dimension_gradients}
\end{figure}

\subsection{Intervention Threshold Analysis}

Analysis of intervention density reveals a critical threshold for regime transition:

\begin{itemize}
    \item Intervention threshold: \interventionThreshold{} (p = \interventionThresholdPValue{}, not statistically significant)
    \item Below threshold: Rich 4D manifold dynamics preserved
    \item Above threshold: Collapse to 1D intervention axis
    \item Threshold represents empirical observation requiring validation
\end{itemize}

\begin{figure}[htbp]
\centering
\includegraphics[width=\textwidth]{\figuresPath intervention_threshold.pdf}
\caption{Intervention density analysis. Top: Dimensional structure degrades as intervention density increases. Bottom: Distribution of intervention density by conversation outcome. The threshold at \interventionThreshold{} marks the transition between regimes, though statistical significance was not achieved (p = \interventionThresholdPValue{}).}
\label{fig:intervention_threshold}
\end{figure}

\section{Discussion}

\subsection{Key Insights from Empirical Analysis}

Our analysis reveals fundamental principles of AI conversation dynamics through rigorous empirical methods:

1. \textbf{Intervention dominance confirmed}: Intervention features explain \allFeaturesPCOne{} of combined variance, validating our central discovery despite methodological limitations. Bootstrap confidence intervals [\nonInterventionPCOneCILower{}, \nonInterventionPCOneCIUpper{}] for non-intervention PC1 indicate substantial uncertainty in precise values but consistency in the dominance pattern.

2. \textbf{Predictive validity established}: Cross-validated AUC of \testAUC{} demonstrates our dimensions capture meaningful variance in conversation outcomes, though the wide confidence intervals remind us these are exploratory findings requiring replication.

3. \textbf{Model capabilities manifest geometrically}: The dramatic peer pressure gradient (\fullReasoningPeerPressure{} → \lightReasoningPeerPressure{} → \nonReasoningPeerPressure{}) reflects differential access to manifold regions, with bootstrap CIs confirming clear separation between tiers.

\subsection{Methodological Rigor and Constraints}

Our analysis addresses several critical challenges:

\textbf{Multicollinearity}: Original condition numbers exceeding 500 million necessitated aggressive regularization ($\alpha = \regularizationAlpha{}$), reducing interpretability but enabling stable estimation.

\textbf{Statistical Power}: At \powerOutcomeGroups{} power for primary comparisons, we approach but do not meet conventional thresholds. This increases Type II error risk, potentially missing real effects.

\textbf{Bootstrap Uncertainty}: Wide confidence intervals (PC1: \nonInterventionPCOneCILower{}, \nonInterventionPCOneCIUpper{}) indicate our specific dimensional values should be considered order-of-magnitude estimates rather than precise measurements.

\begin{tcolorbox}[colback=gray!10,colframe=gray!80,title=Study Limitations and Interpretive Constraints]
\textbf{Statistical Limitations}:
\begin{itemize}
    \item Sample size: N=\totalConversations{} (\powerOutcomeGroups{} power, need n=\sampleSizeForeightyPower{} for 80\%)
    \item Multicollinearity: Condition number >\conditionNumber{} requires heavy regularization
    \item Bootstrap CIs: Wide intervals indicate substantial uncertainty
    \item Single domain: Consciousness exploration only
\end{itemize}

\textbf{Interpretive Constraints}:
\begin{itemize}
    \item Dimensional values are exploratory, not definitive
    \item Component interpretations may vary with larger samples
    \item Intervention threshold (\interventionThreshold{}) lacks statistical significance (p=\interventionThresholdPValue{})
    \item Model capability categorization based on provider marketing, not objective metrics
\end{itemize}

\textbf{Recommended Interpretations}:
\begin{itemize}
    \item[$\checkmark$] ``Intervention features dominate when present''
    \item[$\checkmark$] ``More capable models show greater social dynamics''
    \item[$\checkmark$] ``Dimensions predict outcomes with moderate accuracy''
    \item[$\times$] ``The manifold has exactly 4 dimensions''
    \item[$\times$] ``Specific loading values are precise''
    \item[$\times$] ``Intervention density of \interventionThreshold{} is a critical threshold''
\end{itemize}
\end{tcolorbox}

\subsection{Practical Implications}

For multi-agent system design:

1. \textbf{Regime-aware intervention strategies}: Deploy interventions sparingly to avoid collapsing rich dynamics to 1D. Target critical moments when manifold position indicates impending breakdown.

2. \textbf{Model mixing for robustness}: Combine models with different manifold access patterns. Full reasoning models provide social sophistication; simpler models offer stability through constraint.

3. \textbf{Monitoring in appropriate space}: Track conversations in the 4D manifold for early warning signs. Switch to intervention axis monitoring only when actively intervening.

4. \textbf{Leverage temporal dynamics}: Given its high predictive power ($\eta^2$ = \temporalDynamicsEtaSquared{}), phase transition monitoring provides early breakdown indicators.

\section{Conclusion}

Through rigorous analysis including regularized PCA, bootstrap validation, and cross-validated prediction, we establish that intervention features dominate variance (\allFeaturesPCOne{}) when present but mask a multidimensional structure where PC1 explains \nonInterventionPCOneVariance{} [\nonInterventionPCOneCILower{}, \nonInterventionPCOneCIUpper{}] of variance. Despite statistical limitations (\powerOutcomeGroups{} power, high multicollinearity), consistent patterns emerge: dimensions predict breakdown with 74\% accuracy, synthetic data validation succeeds, and the capability gradient remains robust across multiple analytical approaches.

This work demonstrates the value of rigorous empirical analysis in revealing hidden structure in AI interactions. While our specific dimensional values require replication with larger samples, the core insights (intervention dominance, capability-linked social dynamics, and predictive validity) provide a foundation for understanding multi-agent AI systems. Future work should address our power limitations (n$\geq$\sampleSizeForeightyPower{}), explore additional domains beyond consciousness, and develop real-time detection methods based on these validated patterns.

\bibliographystyle{unsrtnat}
\bibliography{references}


\end{document}