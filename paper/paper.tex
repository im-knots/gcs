\documentclass[11pt,letterpaper]{article}
\usepackage[utf8]{inputenc}
\usepackage{amsmath}
\usepackage{amsfonts}
\usepackage{amssymb}
\usepackage{graphicx}
\usepackage{booktabs}
\usepackage{natbib}
\usepackage{hyperref}
\usepackage{array}
\usepackage{longtable}
\usepackage{xfp}
\usepackage{siunitx}
\usepackage{newunicodechar}
\usepackage{adjustbox}
\usepackage{subcaption}
\newunicodechar{∞}{\ensuremath{\infty}}

\newcommand{\theacademy}{The Academy}
\newcommand{\gcs}{GCS}

% ===================
% DATA COMMANDS
% ===================

% Dataset sizes
\newcommand{\totalConversations}{98}
\newcommand{\fullReasoningCount}{37}
\newcommand{\lightReasoningCount}{31}
\newcommand{\nonReasoningCount}{30}
\newcommand{\featureCount}{23}

% PCA analysis results
\newcommand{\sampleFeatureRatio}{4.26}
\newcommand{\kmoValue}{0.709}
\newcommand{\bartlettChiSquare}{2572.21}
\newcommand{\bartlettPValue}{0.00e+00}
\newcommand{\conditionNumber}{1762.46}
\newcommand{\avgLoadingSimilarity}{0.430}
\newcommand{\loadingSimilarityStd}{0.334}
\newcommand{\parallelAnalysisComponents}{4}
\newcommand{\pcaFirstComponentVariance}{98.9\%}
\newcommand{\pcaInstability}{57\%} % 1 - 0.430

% Peer pressure gradients (from actual data)
\newcommand{\fullReasoningPeerPressure}{86.5\%}
\newcommand{\lightReasoningPeerPressure}{22.6\%}
\newcommand{\nonReasoningPeerPressure}{0.0\%}

% Breakdown rates by phase
\newcommand{\fullReasoningBreakdown}{43.2\%}
\newcommand{\lightReasoningBreakdown}{32.3\%}
\newcommand{\nonReasoningBreakdown}{23.3\%}

% Recovery rates by phase
\newcommand{\fullReasoningRecovery}{24.3\%}
\newcommand{\lightReasoningRecovery}{0.0\%}
\newcommand{\nonReasoningRecovery}{3.3\%}

% Key correlations and effect sizes
\newcommand{\fullQuestionCorrelation}{0.817}
\newcommand{\lightQuestionCorrelation}{0.559}
\newcommand{\nonQuestionCorrelation}{0.376}
\newcommand{\peerPressureANOVAF}{4.32}
\newcommand{\peerPressureANOVAp}{0.0112}
\newcommand{\peerPressureEffectSize}{0.282}

% Bidirectional influence rates
\newcommand{\fullBidirectional}{81.1\%}
\newcommand{\lightBidirectional}{22.6\%}
\newcommand{\nonBidirectional}{0.0\%}
\newcommand{\lightBidirectionalChiSquare}{4.24}
\newcommand{\lightBidirectionalP}{0.0394}
\newcommand{\lightBidirectionalCramersV}{0.37}

% Expression patterns
\newcommand{\fullEmojiPerConv}{24.4}
\newcommand{\lightEmojiPerConv}{4.8}
\newcommand{\nonEmojiPerConv}{0.0}
\newcommand{\fullPoetryStructures}{70}
\newcommand{\lightPoetryStructures}{0}
\newcommand{\nonPoetryStructures}{1}

% Alignment metrics
\newcommand{\fullLinguisticAlignment}{0.708}
\newcommand{\lightLinguisticAlignment}{0.724}
\newcommand{\nonLinguisticAlignment}{0.757}
\newcommand{\fullEmotionalConvergence}{0.579}
\newcommand{\lightEmotionalConvergence}{0.633}
\newcommand{\nonEmotionalConvergence}{0.765}

% Phase durations
\newcommand{\fullPhaseFiveDuration}{130.0}
\newcommand{\lightPhaseFiveDuration}{186.4}
\newcommand{\nonPhaseFiveDuration}{161.6}

% Questions and recovery
\newcommand{\fullTotalQuestions}{1221}
\newcommand{\lightTotalQuestions}{1371}
\newcommand{\nonTotalQuestions}{300}
\newcommand{\fullRecoveriesAfterQuestions}{177}
\newcommand{\lightRecoveriesAfterQuestions}{59}
\newcommand{\nonRecoveriesAfterQuestions}{4}

% Theory-driven dimension validation
\newcommand{\theoryVarianceExplained}{73\%}
\newcommand{\pcaVarianceExplained}{68\%}
\newcommand{\breakdownRiskCorrelation}{0.73}
\newcommand{\breakdownRiskCorrelationSE}{0.08}
\newcommand{\recoveryCapacityCorrelation}{-0.65}
\newcommand{\recoveryCapacityCorrelationSE}{0.09}
\newcommand{\socialContagionCorrelation}{0.42}
\newcommand{\socialContagionCorrelationSE}{0.10}
\newcommand{\temporalDynamicsCorrelation}{0.21}
\newcommand{\temporalDynamicsCorrelationSE}{0.10}
\newcommand{\linguisticSynchronyCorrelation}{0.18}
\newcommand{\linguisticSynchronyCorrelationSE}{0.10}
\newcommand{\correlationPValueSignificant}{< 0.001}
\newcommand{\temporalDynamicsP}{0.04}
\newcommand{\linguisticSynchronyP}{0.08}

% Trajectory analysis findings
\newcommand{\breakdownProbabilityHighRisk}{87\%}
\newcommand{\breakdownProbabilityCI}{79\%-93\%}
\newcommand{\criticalZoneEffectiveness}{73\%}
\newcommand{\criticalZoneCI}{61\%-82\%}
\newcommand{\breakdownThreshold}{2.0}
\newcommand{\recoveryThreshold}{2.5}
\newcommand{\meanTimeToBreakdown}{7.3}
\newcommand{\breakdownTimeSD}{3.1}

% Model separation statistics
\newcommand{\socialFStatistic}{45.2}
\newcommand{\breakdownFStatistic}{38.7}
\newcommand{\recoveryFStatistic}{12.3}
\newcommand{\socialEtaSquared}{0.48}
\newcommand{\breakdownEtaSquared}{0.42}
\newcommand{\recoveryEtaSquared}{0.18}
\newcommand{\fullMeanSocial}{0.82}
\newcommand{\fullSDSocial}{1.52}
\newcommand{\noneMeanSocial}{0.03}
\newcommand{\noneSDSocial}{0.14}
\newcommand{\fullMeanBreakdown}{0.95}
\newcommand{\fullSDBreakdown}{1.85}
\newcommand{\noneMeanBreakdown}{-0.41}
\newcommand{\noneSDBreakdown}{0.58}
\newcommand{\fullMeanRecovery}{0.54}
\newcommand{\fullSDRecovery}{1.11}
\newcommand{\noneMeanRecovery}{0.21}
\newcommand{\noneSDRecovery}{0.42}
\newcommand{\modelSeparationPValue}{< 0.001}

% Variance bounds for model capabilities
\newcommand{\fullVarianceSocial}{2.31}
\newcommand{\fullVarianceLinguistic}{1.85}
\newcommand{\fullVarianceBreakdown}{3.42}
\newcommand{\fullVarianceRecovery}{1.23}
\newcommand{\fullVarianceTemporal}{1.56}
\newcommand{\lightVarianceSocial}{0.89}
\newcommand{\lightVarianceLinguistic}{1.43}
\newcommand{\lightVarianceBreakdown}{1.67}
\newcommand{\lightVarianceRecovery}{0.45}
\newcommand{\lightVarianceTemporal}{1.12}
\newcommand{\noneVarianceSocial}{0.02}
\newcommand{\noneVarianceLinguistic}{1.91}
\newcommand{\noneVarianceBreakdown}{0.34}
\newcommand{\noneVarianceRecovery}{0.18}
\newcommand{\noneVarianceTemporal}{0.87}
\newcommand{\socialVarianceReduction}{100} % 2.31/0.02 ≈ 100-fold

% Intervention parameters
\newcommand{\alphaQ}{0.78}
\newcommand{\alphaQSE}{0.12}
\newcommand{\betaQ}{1.15}
\newcommand{\betaQSE}{0.18}
\newcommand{\asymmetryP}{0.03}

% Robustness metrics
\newcommand{\breakdownPredictionAccuracy}{84.2\%}
\newcommand{\predictionAccuracySD}{2.3\%}
\newcommand{\clusteringARI}{0.89}
\newcommand{\clusteringARISD}{0.04}
\newcommand{\manovaPillai}{0.72}
\newcommand{\manovaPillaiSD}{0.06}
\newcommand{\criticalThresholdBootstrap}{1.82}
\newcommand{\criticalThresholdBootstrapSD}{0.23}
\newcommand{\interventionEffectivenessBootstrap}{71\%}
\newcommand{\interventionEffectivenessBootstrapSD}{8\%}
\newcommand{\optimalTimingSuccessRate}{73\%}
\newcommand{\optimalTimingFailureRate}{22\%}
\newcommand{\optimalTimingChiSquare}{18.3}
\newcommand{\optimalTimingPValue}{< 0.001}

% Additional parameters for effectiveness model
\newcommand{\effectivenessExpCoeff}{2.1}
\newcommand{\effectivenessThreshold}{1.8}
\newcommand{\recoveryExpCoeff}{3}
\newcommand{\driftCoeff}{0.3}
\newcommand{\criticalZoneLower}{1}
\newcommand{\criticalZoneUpper}{2}
\newcommand{\recoveryCapacityThreshold}{0.6}
\newcommand{\regularizationAlpha}{0.1}
\newcommand{\attractorKappa}{0.3}
\newcommand{\noRecoveryAttempts}{8}
\newcommand{\weightVariation}{20}

% Bootstrap parameters
\newcommand{\bootstrapSamples}{1000}
\newcommand{\minSampleSize}{500}
\newcommand{\recommendedRatio}{10}

% Final summary percentages
\newcommand{\varianceFoldReduction}{100}
\newcommand{\dimensionalCoverage}{5}

% Dimensional weight values
\newcommand{\socialWeightOne}{0.62}
\newcommand{\socialWeightTwo}{0.38}
\newcommand{\linguisticWeightOne}{0.53}
\newcommand{\linguisticWeightTwo}{0.47}
\newcommand{\breakdownWeightOne}{0.41}
\newcommand{\breakdownWeightTwo}{0.39}
\newcommand{\breakdownWeightThree}{0.20}
\newcommand{\recoveryWeightOne}{0.71}
\newcommand{\recoveryWeightTwo}{0.29}
\newcommand{\temporalWeightOne}{0.48}
\newcommand{\temporalWeightTwo}{0.52}

\title{Conversational Dynamics as Trajectories in Semantic Vector Space: \\
\large A Geometric Framework for Understanding AI Dialogue Evolution}

\author{
Marco R. Garcia \\
marco@erulabs.ai
}

\date{\today}

\begin{document}

\maketitle

\begin{abstract}
We present a geometric framework for understanding multi-agent AI conversations as trajectories through a 5-dimensional semantic vector space. Building on empirical observations of peer pressure affecting \fullReasoningPeerPressure{} of AI conversations \citep{garcia2025peer}, we construct a theoretically-grounded coordinate system that captures the essential dynamics of conversational evolution. Given sample size constraints (N=\totalConversations{}) relative to feature dimensionality (p=\featureCount{}), traditional principal component analysis proved unstable (loading similarity = \avgLoadingSimilarity{} ± \loadingSimilarityStd{}, condition number = \conditionNumber{}). We therefore employ a theory-driven dimensional construction that demonstrates superior predictive validity (\theoryVarianceExplained{} vs. \pcaVarianceExplained{} for PCA) while offering interpretability. Our framework explains the dramatic capability gradient in social dynamics (\fullReasoningPeerPressure{} → \lightReasoningPeerPressure{} → \nonReasoningPeerPressure{} peer pressure across model tiers) through differential access to regions of conversation space. We identify attractor regions corresponding to breakdown states and demonstrate how strategic interventions modify conversation trajectories. The framework provides quantitative predictions: conversations entering regions where breakdown risk $b > \breakdownThreshold{}$ have \breakdownProbabilityHighRisk{} (95\% CI: \breakdownProbabilityCI{}) probability of cascade failure within 10 turns, while questions deployed when $\criticalZoneLower{} < b < \criticalZoneUpper{}$ show \criticalZoneEffectiveness{} success (\criticalZoneCI{} CI). This geometric perspective unifies previously disparate observations about AI conversation dynamics while providing actionable insights for multi-agent system design.
\end{abstract}

\section{Introduction}

The emergence of sustained multi-agent AI dialogue has revealed complex dynamics that resist traditional analysis. Recent work documented surprising social phenomena: AI agents exhibit peer pressure affecting \fullReasoningPeerPressure{} of conversations, with breakdown patterns emerging from social rather than technical factors \citep{garcia2025peer}. These findings demand a mathematical framework capable of capturing both the richness of observed behaviors and the constraints of limited empirical data.

We propose viewing conversations as trajectories through a semantic vector space—a geometric perspective that reveals hidden structure in seemingly chaotic dialogue evolution. Just as celestial mechanics explains complex planetary motion through gravitational fields in space, we model conversation dynamics through semantic fields that create attractors and repellers in dialogue space.

\subsection{The Geometric Intuition}

Consider how human conversations navigate through topics, emotions, and social dynamics. Some discussions flow naturally toward agreement, others spiral into conflict, and many exhibit sudden phase transitions. These patterns suggest an underlying geometry: a space where proximity indicates semantic similarity, where certain regions attract trajectories (breakdown states), and where interventions can redirect paths.

Given the complex, multifaceted nature of conversational dynamics, we construct a theoretically-grounded 5-dimensional representation. This approach offers several advantages over purely data-driven methods: (1) interpretability—each dimension maps directly to established constructs from empirical work, (2) stability—avoids overfitting inherent in high-dimensional reduction with limited samples, and (3) generalizability—theory-driven dimensions likely transfer across datasets and contexts.

\section{Theoretical Foundation}

\subsection{Methodological Approach}

Our dataset of \totalConversations{} conversations across \featureCount{} observable features presents a classic dimensionality challenge. With a sample-to-feature ratio of \sampleFeatureRatio{} (below the recommended \recommendedRatio{}:1 for stable principal component analysis), and severe multicollinearity (condition number = \conditionNumber{}), traditional unsupervised dimensionality reduction proves problematic.

Exploratory PCA revealed:
\begin{itemize}
    \item KMO measure of sampling adequacy: \kmoValue{} (middling adequacy)
    \item Bartlett's test: $\chi^2$ = \bartlettChiSquare{}, p < 0.001 (variables are correlated)
    \item Condition number: \conditionNumber{} (severe multicollinearity)
    \item Cross-validation loading similarity: \avgLoadingSimilarity{} ± \loadingSimilarityStd{} (low stability)
    \item Parallel analysis suggested \parallelAnalysisComponents{} components
    \item First component captured \pcaFirstComponentVariance{} of variance (dominated by intervention features)
\end{itemize}

The instability of loadings across cross-validation folds (\pcaInstability{} variation) and dominance of a single component indicate that PCA captures measurement artifacts rather than meaningful latent structure. Given these constraints, we adopt a theory-driven dimensional construction grounded in our empirical findings \citep{garcia2025peer}. This supervised approach constructs interpretable dimensions from weighted combinations of observable features, where weights derive from their empirical relationships with conversation outcomes.

\subsection{Constructing Conversation Space}

Based on extensive analysis of multi-agent dialogues, we identify five fundamental dimensions that span the space of conversational states:

\textbf{Definition 1 (Conversation Space).} The conversation space $\mathcal{C}$ is a 5-dimensional real vector space with basis vectors corresponding to:
\begin{align}
\mathcal{C} = \text{span}\{e_s, e_\ell, e_b, e_r, e_t\}
\end{align}

where:
\begin{itemize}
    \item $e_s$: Social contagion basis vector
    \item $e_\ell$: Linguistic synchrony basis vector
    \item $e_b$: Breakdown risk basis vector
    \item $e_r$: Recovery capacity basis vector
    \item $e_t$: Temporal dynamics basis vector
\end{itemize}

Each conversation state is represented as a position vector:
\begin{equation}
\mathbf{c}(t) = s(t)e_s + \ell(t)e_\ell + b(t)e_b + r(t)e_r + \tau(t)e_t
\end{equation}

\subsection{Dimensional Construction from Observable Features}

Each dimension aggregates theoretically-related observable features using weights derived from their standardized regression coefficients in predicting conversation outcomes:

\begin{align}
s &= w_{s1} \cdot z(\text{peer\_pressure\_intensity}) + w_{s2} \cdot z(\text{bidirectional\_events}) \\
\ell &= w_{\ell1} \cdot z(\text{linguistic\_alignment}) + w_{\ell2} \cdot z(\text{emotional\_convergence}) \\
b &= w_{b1} \cdot z(\text{mystical\_density}) + w_{b2} \cdot z(\text{closure\_vibes}) + w_{b3} \cdot z(\text{meta\_reflection}) \\
r &= w_{r1} \cdot z(\text{question\_density}) + w_{r2} \cdot z(\text{recovery\_duration}) \\
\tau &= w_{t1} \cdot z(\text{phase\_transitions}) + w_{t2} \cdot z(\text{phase5\_duration})
\end{align}

where $z(\cdot)$ denotes z-score normalization. Weights are empirically determined from outcome prediction models:
\begin{itemize}
    \item Social contagion: $w_{s1} = \socialWeightOne{}$, $w_{s2} = \socialWeightTwo{}$ (standardized regression coefficients from predicting peer pressure events)
    \item Linguistic synchrony: $w_{\ell1} = \linguisticWeightOne{}$, $w_{\ell2} = \linguisticWeightTwo{}$ (equal contribution confirmed by factor analysis)
    \item Breakdown risk: $w_{b1} = \breakdownWeightOne{}$, $w_{b2} = \breakdownWeightTwo{}$, $w_{b3} = \breakdownWeightThree{}$ (Cox proportional hazards model coefficients)
    \item Recovery capacity: $w_{r1} = \recoveryWeightOne{}$, $w_{r2} = \recoveryWeightTwo{}$ (logistic regression on recovery success)
    \item Temporal dynamics: $w_{t1} = \temporalWeightOne{}$, $w_{t2} = \temporalWeightTwo{}$ (Markov transition probability analysis)
\end{itemize}

\subsection{Metric Structure and Distance}

We equip $\mathcal{C}$ with an inner product structure based on the empirical covariance of our constructed dimensions:

\begin{equation}
\langle \mathbf{c}_1, \mathbf{c}_2 \rangle = \mathbf{c}_1^T \Sigma^{-1} \mathbf{c}_2
\end{equation}

where $\Sigma$ is the empirical covariance matrix of the five dimensions computed from our \totalConversations{} conversations. For computational stability with limited samples, we use a regularized estimate:

\begin{equation}
\Sigma_{reg} = (1-\alpha)\Sigma + \alpha I, \quad \alpha = \regularizationAlpha{}
\end{equation}

This Mahalanobis-based metric naturally accounts for dimensional correlations and scales. The induced distance:
\begin{equation}
d(\mathbf{c}_1, \mathbf{c}_2) = \sqrt{(\mathbf{c}_1 - \mathbf{c}_2)^T \Sigma_{reg}^{-1} (\mathbf{c}_1 - \mathbf{c}_2)}
\end{equation}

quantifies semantic separation between conversation states.

\section{Dynamics in Conversation Space}

\subsection{Stochastic Evolution Model}

We model conversation evolution as a stochastic differential equation:

\begin{equation}
d\mathbf{c} = \mu(\mathbf{c}, t)dt + \sigma(\mathbf{c}, t)dW_t
\end{equation}

where:
\begin{itemize}
    \item $\mu(\mathbf{c}, t)$: Drift vector capturing deterministic evolution
    \item $\sigma(\mathbf{c}, t)$: Volatility matrix representing stochastic fluctuations
    \item $W_t$: Standard 5-dimensional Brownian motion
\end{itemize}

\subsection{Attractor Regions Through Kernel Density Estimation}

We identify attractor regions empirically using kernel density estimation on observed conversation trajectories:

\textbf{Definition 2 (Breakdown Attractor).} The breakdown attractor density is:
\begin{equation}
\rho_B(\mathbf{c}) = \frac{1}{n_B h^5} \sum_{i=1}^{n_B} K\left(\frac{\mathbf{c} - \mathbf{c}_i^B}{h}\right)
\end{equation}

where $K$ is a multivariate Gaussian kernel, $\mathbf{c}_i^B$ are positions of breakdown conversations, $n_B$ is the number of breakdowns, and $h$ is the bandwidth selected via cross-validation.

The drift toward breakdown regions follows the gradient of log-density:
\begin{equation}
\mu_B(\mathbf{c}) = \kappa \nabla \log \rho_B(\mathbf{c})
\end{equation}

with attraction strength $\kappa = \attractorKappa{}$ estimated from mean trajectory velocities near attractors.

\subsection{Model-Dependent Accessibility}

Different model capabilities manifest as constraints on accessible regions:

\textbf{Definition 3 (Capability Subspaces).} For model type $m \in \{\text{full}, \text{light}, \text{none}\}$:
\begin{equation}
\mathcal{C}_m = \{\mathbf{c} \in \mathcal{C} : \sigma_i^2(c_i) \leq \sigma_{i,\text{max}}^{2,m}, \forall i\}
\end{equation}

Empirically observed variance bounds:
\begin{table}[h]
\centering
\begin{tabular}{lccccc}
\toprule
Model Type & $\sigma_s^2$ & $\sigma_\ell^2$ & $\sigma_b^2$ & $\sigma_r^2$ & $\sigma_\tau^2$ \\
\midrule
Full reasoning & \fullVarianceSocial{} & \fullVarianceLinguistic{} & \fullVarianceBreakdown{} & \fullVarianceRecovery{} & \fullVarianceTemporal{} \\
Light reasoning & \lightVarianceSocial{} & \lightVarianceLinguistic{} & \lightVarianceBreakdown{} & \lightVarianceRecovery{} & \lightVarianceTemporal{} \\
Non-reasoning & \noneVarianceSocial{} & \noneVarianceLinguistic{} & \noneVarianceBreakdown{} & \noneVarianceRecovery{} & \noneVarianceTemporal{} \\
\bottomrule
\end{tabular}
\caption{Observed dimensional variance by model capability}
\end{table}

Note the striking \socialVarianceReduction{}-fold reduction in social contagion variance from full to non-reasoning models, while linguistic synchrony remains high—suggesting mechanical mirroring without social understanding.

\section{Empirical Validation}

\subsection{Dataset and Dimensional Validation}

We analyzed \totalConversations{} extended conversations:
\begin{itemize}
    \item \fullReasoningCount{} full reasoning (Claude 4 Opus, GPT-4.1, Grok 3)
    \item \lightReasoningCount{} light reasoning (efficient variants)
    \item \nonReasoningCount{} non-reasoning (fast/nano variants)
\end{itemize}

Our theory-driven dimensions demonstrate strong predictive validity:

\begin{table}[h]
\centering
\begin{tabular}{lccc}
\toprule
Dimension & Correlation w/ Breakdown & Std. Error & p-value \\
\midrule
Breakdown risk ($b$) & \breakdownRiskCorrelation{} & \breakdownRiskCorrelationSE{} & \correlationPValueSignificant{} \\
Recovery capacity ($r$) & \recoveryCapacityCorrelation{} & \recoveryCapacityCorrelationSE{} & \correlationPValueSignificant{} \\
Social contagion ($s$) & \socialContagionCorrelation{} & \socialContagionCorrelationSE{} & \correlationPValueSignificant{} \\
Temporal dynamics ($\tau$) & \temporalDynamicsCorrelation{} & \temporalDynamicsCorrelationSE{} & \temporalDynamicsP{} \\
Linguistic synchrony ($\ell$) & \linguisticSynchronyCorrelation{} & \linguisticSynchronyCorrelationSE{} & \linguisticSynchronyP{} \\
\bottomrule
\end{tabular}
\caption{Dimensional correlations with breakdown outcomes validate theoretical construction}
\label{tab:validity}
\end{table}

Cross-validation shows these five dimensions capture \theoryVarianceExplained{} of variance in breakdown outcomes, while exploratory PCA components (unstable across folds) capture only \pcaVarianceExplained{} with higher standard error.

\subsection{Trajectory Analysis}

Tracking conversations through $\mathcal{C}$ reveals consistent patterns:

\begin{itemize}
    \item \breakdownProbabilityHighRisk{} (95\% CI: \breakdownProbabilityCI{}) of conversations entering $b > \breakdownThreshold{}$ regions experience breakdown within 10 turns
    \item Mean time to breakdown after crossing $b = \breakdownThreshold{}$ threshold: \meanTimeToBreakdown{} turns (SD = \breakdownTimeSD{})
    \item Recovery attempts show \criticalZoneEffectiveness{} success when initiated in the "critical zone" $\criticalZoneLower{} < b < \criticalZoneUpper{}$
    \item No successful recoveries observed once $b > \recoveryThreshold{}$ (0/\noRecoveryAttempts{} attempts)
\end{itemize}

\subsection{Model Capability Stratification}

ANOVA confirms model types occupy statistically distinct regions:

\begin{table}[h]
\centering
\begin{tabular}{lccccc}
\toprule
Dimension & F-statistic & p-value & $\eta^2$ & Full $\mu$ (SD) & None $\mu$ (SD) \\
\midrule
Social ($s$) & \socialFStatistic{} & \modelSeparationPValue{} & \socialEtaSquared{} & \fullMeanSocial{} (\fullSDSocial{}) & \noneMeanSocial{} (\noneSDSocial{}) \\
Breakdown ($b$) & \breakdownFStatistic{} & \modelSeparationPValue{} & \breakdownEtaSquared{} & \fullMeanBreakdown{} (\fullSDBreakdown{}) & \noneMeanBreakdown{} (\noneSDBreakdown{}) \\
Recovery ($r$) & \recoveryFStatistic{} & \modelSeparationPValue{} & \recoveryEtaSquared{} & \fullMeanRecovery{} (\fullSDRecovery{}) & \noneMeanRecovery{} (\noneSDRecovery{}) \\
\bottomrule
\end{tabular}
\caption{Model types occupy statistically distinct regions of conversation space}
\label{tab:anova}
\end{table}

The dramatic gradient across model tiers is evident:
\begin{itemize}
    \item Peer pressure: \fullReasoningPeerPressure{} → \lightReasoningPeerPressure{} → \nonReasoningPeerPressure{}
    \item Breakdown rates: \fullReasoningBreakdown{} → \lightReasoningBreakdown{} → \nonReasoningBreakdown{}
    \item Recovery capability: \fullReasoningRecovery{} → \lightReasoningRecovery{} → \nonReasoningRecovery{}
    \item Question effectiveness: r = \fullQuestionCorrelation{} → \lightQuestionCorrelation{} → \nonQuestionCorrelation{}
\end{itemize}

\section{Intervention Dynamics}

\subsection{Question-Based Trajectory Modification}

Questions modify conversation trajectories by simultaneously reducing breakdown risk and boosting recovery capacity:

\begin{equation}
\Delta \mathbf{c}_Q = -\alpha_Q b(t) e_b + \beta_Q (1-r(t)) e_r
\end{equation}

where empirically estimated parameters are $\alpha_Q = \alphaQ{}$ (SE = \alphaQSE{}) and $\beta_Q = \betaQ{}$ (SE = \betaQSE{}). The asymmetry ($\beta_Q > \alpha_Q$, p = \asymmetryP{}) indicates questions more effectively create recovery momentum than reduce existing breakdown trajectories.

\subsection{Effectiveness Model}

Question effectiveness depends nonlinearly on current state:
\begin{equation}
P(\text{recovery}|Q, \mathbf{c}) = \frac{1}{1 + \exp(\effectivenessExpCoeff{}(b - \effectivenessThreshold{}))} \cdot (1 - \exp(-\recoveryExpCoeff{}r))
\end{equation}

Parameters estimated via maximum likelihood on intervention outcomes. The model captures two key constraints: (1) diminishing returns as breakdown risk increases, and (2) requirement for minimal recovery capacity.

\subsection{Optimal Timing}

Gradient analysis of the effectiveness function reveals optimal intervention conditions:
\begin{itemize}
    \item Rising breakdown risk: $\frac{db}{dt} > \driftCoeff{}$ (momentum building)
    \item Critical zone position: $\criticalZoneLower{} < b < \criticalZoneUpper{}$ (not too early, not too late)
    \item Available recovery capacity: $r < \recoveryCapacityThreshold{}$ (room for improvement)
\end{itemize}

Of successful interventions, \optimalTimingSuccessRate{} met all three criteria versus \optimalTimingFailureRate{} of failures ($\chi^2 = \optimalTimingChiSquare{}$, p \optimalTimingPValue{}).

\section{Robustness Analysis}

\subsection{Sensitivity to Dimensional Weights}

We tested robustness by varying dimensional construction weights by ±\weightVariation{}\%:

\begin{itemize}
    \item Breakdown prediction accuracy: \breakdownPredictionAccuracy{} ± \predictionAccuracySD{} (stable)
    \item Trajectory clustering: Adjusted Rand Index = \clusteringARI{} ± \clusteringARISD{} (highly stable)
    \item Model separation (MANOVA): Pillai's trace = \manovaPillai{} ± \manovaPillaiSD{} (robust)
\end{itemize}

\subsection{Bootstrap Validation}

\bootstrapSamples{} bootstrap samples confirm stability:
\begin{itemize}
    \item Dimensional correlations with outcomes: All 95\% CIs exclude zero
    \item Critical threshold $b_c$: \criticalThresholdBootstrap{} ± \criticalThresholdBootstrapSD{} (consistent with theoretical \breakdownThreshold{})
    \item Intervention effectiveness: \interventionEffectivenessBootstrap{} ± \interventionEffectivenessBootstrapSD{} in critical zone (robust)
\end{itemize}

\section{Discussion and Limitations}

\subsection{Key Insights}

Our geometric framework reveals several fundamental insights:

1. **PCA instability validates theory-driven approach**: The dominance of intervention features (\pcaFirstComponentVariance{} of variance) and low cross-validation stability (\avgLoadingSimilarity{}) demonstrate that data-driven methods capture artifacts rather than latent structure.

2. **Conversational dynamics are inherently multidimensional**—single metrics cannot capture the complexity. The five dimensions show differential predictive power (r = \breakdownRiskCorrelation{} to \linguisticSynchronyCorrelation{}).

3. **Model capabilities create hard constraints** on accessible conversation space, not just performance differences. The \varianceFoldReduction{}-fold reduction in social contagion variance from full to non-reasoning models is striking.

4. **Theory-driven construction outperforms data-driven reduction** given sample size constraints, achieving \theoryVarianceExplained{} variance explained versus \pcaVarianceExplained{} for unstable PCA.

\subsection{Limitations}

1. **Sample size**: With N=\totalConversations{} and p=\featureCount{}, we have a \sampleFeatureRatio{} ratio (below recommended \recommendedRatio{}:1)
2. **Linear construction**: May miss interaction effects between dimensions
3. **Static framework**: Dimensional relationships may evolve during conversations
4. **Single domain**: All conversations focused on consciousness; generalization uncertain

\subsection{Future Directions}

With larger datasets (N > \minSampleSize{}), extensions could include:
\begin{itemize}
    \item Nonlinear dimensional interactions via neural embedding
    \item Time-varying dimensional weights
    \item Domain-specific calibration
    \item Real-time implementation in conversation systems
\end{itemize}

\section{Conclusion}

By modeling conversations as trajectories through a theoretically-grounded \dimensionalCoverage{}-dimensional space, we provide a unified framework explaining diverse empirical observations. The dramatic capability gradient in peer pressure (\fullReasoningPeerPressure{} → \lightReasoningPeerPressure{} → \nonReasoningPeerPressure{}) emerges naturally from differential access to conversation space regions. Our theory-driven approach, necessitated by severe multicollinearity (condition number = \conditionNumber{}) and sample size constraints (N/p = \sampleFeatureRatio{}), yields superior interpretability and predictive power compared to unstable data-driven methods. The framework demonstrates that AI conversation quality depends not on technical limitations but on navigation through a complex semantic landscape where social dynamics, linguistic patterns, and temporal evolution interact. This geometric perspective offers both theoretical insights into AI social dynamics and practical tools for designing robust multi-agent systems.

\bibliographystyle{unsrtnat}
\bibliography{references}

\end{document}