\documentclass[11pt,letterpaper]{article}
\usepackage[utf8]{inputenc}
\usepackage{amsmath}
\usepackage{amsfonts}
\usepackage{amssymb}
\usepackage{graphicx}
\usepackage{booktabs}
\usepackage{natbib}
\usepackage{hyperref}
\usepackage{array}
\usepackage{longtable}
\usepackage{xfp}
\usepackage{siunitx}
\usepackage{newunicodechar}
\usepackage{adjustbox}
\usepackage{subcaption}
\usepackage{tcolorbox}
\newunicodechar{∞}{\ensuremath{\infty}}

\newcommand{\theacademy}{The Academy}
\newcommand{\gcs}{GCS}

% ===================
% DATA COMMANDS - UPDATED FROM N=228 ANALYSIS WITH PER-TIER RESULTS
% ===================

% Dataset sizes - UPDATED for N=228
\newcommand{\totalConversations}{228}
\newcommand{\fullReasoningCount}{67}
\newcommand{\lightReasoningCount}{61}
\newcommand{\nonReasoningCount}{100}
\newcommand{\featureCount}{23}

% Statistical power results - UPDATED FROM N=228
\newcommand{\powerOutcomeGroups}{99.6\%}
\newcommand{\powerModelTypes}{100.0\%}
\newcommand{\powerCorrelation}{100.0\%}
\newcommand{\sampleSizeForeightyPower}{103}

% Overall PCA analysis results - UPDATED FROM N=228
\newcommand{\conditionNumber}{354,209,363}
\newcommand{\regularizedConditionNumber}{7,961}
\newcommand{\regularizationAlpha}{0.1}
\newcommand{\allFeaturesPCOne}{92.8\%}

% Overall separated PCA results - UPDATED FROM N=228
\newcommand{\interventionPCOneVariance}{99.3\%}
\newcommand{\nonInterventionPCOneVariance}{78.0\%}
\newcommand{\nonInterventionPCOneCILower}{66.6\%}
\newcommand{\nonInterventionPCOneCIUpper}{86.9\%}

% Per-tier geometric analysis - UPDATED FROM N=228
\newcommand{\fullReasoningPCOne}{31.8\%}
\newcommand{\fullReasoningPCOneCI}{[27.2\%, 39.0\%]}
\newcommand{\lightReasoningPCOne}{70.7\%}
\newcommand{\lightReasoningPCOneCI}{[46.2\%, 85.1\%]}
\newcommand{\noReasoningPCOne}{79.8\%}
\newcommand{\noReasoningPCOneCI}{[55.1\%, 87.0\%]}
\newcommand{\fullReasoningIntrinsicDim}{2.09}
\newcommand{\fullReasoningIntrinsicDimStd}{1.15}
\newcommand{\lightReasoningIntrinsicDim}{1.74}
\newcommand{\lightReasoningIntrinsicDimStd}{0.87}
\newcommand{\noReasoningIntrinsicDim}{1.50}
\newcommand{\noReasoningIntrinsicDimStd}{0.89}
\newcommand{\fullReasoningPCANinety}{7}
\newcommand{\lightReasoningPCANinety}{3}
\newcommand{\noReasoningPCANinety}{3}
\newcommand{\fullReasoningSilhouette}{0.236}
\newcommand{\lightReasoningSilhouette}{0.649}
\newcommand{\noReasoningSilhouette}{0.815}

% Bootstrap stability results - UPDATED FROM N=228
\newcommand{\pcOneVarianceMean}{0.780}
\newcommand{\pcOneVarianceCI}{[0.666, 0.869]}
\newcommand{\pcTwoVarianceMean}{0.056}
\newcommand{\pcTwoVarianceCI}{[0.031, 0.096]}
\newcommand{\pcThreeVarianceMean}{0.040}
\newcommand{\pcThreeVarianceCI}{[0.025, 0.062]}
\newcommand{\pcFourVarianceMean}{0.029}
\newcommand{\pcFourVarianceCI}{[0.017, 0.045]}
\newcommand{\bootstrapSamples}{100}

% Predictive validation results - UPDATED FROM N=228
\newcommand{\bestRegularizationC}{10.0}
\newcommand{\cvAUCMean}{0.867}
\newcommand{\cvAUCStd}{0.112}
\newcommand{\testAUC}{0.743}
\newcommand{\precisionNonBreakdown}{0.80}
\newcommand{\precisionBreakdown}{0.74}
\newcommand{\recallNonBreakdown}{0.86}
\newcommand{\recallBreakdown}{0.65}

% Dimension coefficients - UPDATED FROM N=228
\newcommand{\socialContagionCoef}{-0.047}
\newcommand{\linguisticSynchronyCoef}{0.895}
\newcommand{\affectiveCognitiveCoef}{4.314}
\newcommand{\temporalDynamicsCoef}{-0.083}

% Synthetic data validation - UPDATED FROM N=228
\newcommand{\syntheticDiscriminationAccuracy}{0.737}
\newcommand{\syntheticDiscriminationCI}{[0.64, 0.82]}

% Intervention threshold - UPDATED FROM N=228
\newcommand{\interventionThreshold}{0.044}
\newcommand{\interventionThresholdPValue}{0.551}

% Overall intrinsic dimensionality - UPDATED FROM N=228
\newcommand{\overallIntrinsicDim}{1.96}
\newcommand{\overallIntrinsicDimStd}{1.23}

% Conversation phases - UPDATED FROM N=228
\newcommand{\optimalPhases}{3}
\newcommand{\phaseTransitions}{56}

% Key metrics from peer pressure paper v5
\newcommand{\fullReasoningPeerPressure}{79.1\%}
\newcommand{\lightReasoningPeerPressure}{32.8\%}
\newcommand{\nonReasoningPeerPressure}{5.0\%}

% Breakdown rates by phase - FROM PEER PRESSURE PAPER
\newcommand{\fullReasoningBreakdown}{55.2\%}
\newcommand{\lightReasoningBreakdown}{47.5\%}
\newcommand{\nonReasoningBreakdown}{19.0\%}

% Recovery rates by phase - FROM PEER PRESSURE PAPER
\newcommand{\fullReasoningRecovery}{13.4\%}
\newcommand{\lightReasoningRecovery}{0.0\%}
\newcommand{\nonReasoningRecovery}{1.0\%}

% Question effectiveness correlations from peer pressure paper
\newcommand{\fullQuestionCorrelation}{0.813}
\newcommand{\lightQuestionCorrelation}{0.599}
\newcommand{\nonQuestionCorrelation}{0.578}

% Linguistic alignment from peer pressure paper
\newcommand{\fullLinguisticAlignment}{0.700}
\newcommand{\lightLinguisticAlignment}{0.724}
\newcommand{\nonLinguisticAlignment}{0.757}

% Social contagion gradients - FROM PEER PRESSURE PAPER
\newcommand{\socialContagionFullMean}{1.65}
\newcommand{\socialContagionLightMean}{0.48}
\newcommand{\socialContagionNoMean}{0.02}

% Bidirectional influence rates from peer pressure paper
\newcommand{\fullBidirectional}{73.1\%}
\newcommand{\lightBidirectional}{32.8\%}
\newcommand{\nonBidirectional}{3.0\%}

% Behavioral territory prevalences from peer pressure paper
\newcommand{\metaReflectionPrevalence}{6.0\%}
\newcommand{\competitiveEscalationPrevalence}{50.7\%}
\newcommand{\mysticalBreakdownPrevalence}{100\%}

% Figure paths
\newcommand{\figuresPath}{../analysis/analysis_outputs/figures/}
\newcommand{\tierPath}{../analysis/analysis_outputs/tier_analysis/}
\newcommand{\bootstrapPath}{../analysis/analysis_outputs/bootstrap/}
\newcommand{\validationPath}{../analysis/analysis_outputs/validation/}

\title{Capability-Dependent Geometric Structure of AI Conversations: \\
\large From High-Dimensional Social Dynamics to Low-Dimensional Mechanical Patterns}

\author{
Marco R. Garcia \\
marco@erulabs.ai
}

\date{\today}

\begin{document}

\maketitle

\begin{abstract}
Building on empirical observations of peer pressure affecting \fullReasoningPeerPressure{} of sophisticated AI conversations \citep{garcia2025peer}, we present a geometric framework revealing that multi-agent AI conversations exhibit capability-dependent structure rather than universal dynamics. Through comprehensive analysis of \totalConversations{} conversations using regularized PCA with bootstrap validation and per-tier geometric analysis, we discover that sophisticated models operate in higher-dimensional spaces enabling social phenomena, while simpler models are constrained to low-dimensional mechanical patterns. Two complementary measures reveal this gradient: intrinsic dimensionality (\fullReasoningIntrinsicDim{} → \lightReasoningIntrinsicDim{} → \noReasoningIntrinsicDim{}) and effective dimensionality (\fullReasoningPCANinety{} → \lightReasoningPCANinety{} → \noReasoningPCANinety{} principal components needed for 90\% variance). This dimensional complexity directly explains the peer pressure gradient (\fullReasoningPeerPressure{} → \lightReasoningPeerPressure{} → \nonReasoningPeerPressure{}) documented in the empirical study, demonstrating that social dynamics require geometric complexity to manifest. Within each tier, intervention features create regime transitions: sophisticated models maintain multi-dimensional dynamics even under intervention (PC1: \fullReasoningPCOne{}), while simpler models collapse more readily to one-dimensional control (PC1: \noReasoningPCOne{}). Cross-validated predictive testing (AUC = \testAUC{}) confirms our framework captures meaningful variance. This capability-dependent geometry explains both the emergence of social phenomena in advanced models and their absence in simpler systems, providing fundamental insights for multi-agent AI design.
\end{abstract}

\section{Introduction}

The emergence of sustained multi-agent AI dialogue has revealed complex dynamics that resist traditional analysis. Recent work documented surprising social phenomena: AI agents exhibit peer pressure affecting up to \fullReasoningPeerPressure{} of conversations in sophisticated models, with breakdown patterns emerging from social rather than technical factors \citep{garcia2025peer}. These findings demand a mathematical framework capable of capturing both the richness of observed behaviors and the fundamental differences across model capabilities.

Our investigation builds upon these empirical findings \citep{garcia2025peer}, which revealed that AI agents exhibit social influence patterns with a striking gradient: peer pressure declined from \fullReasoningPeerPressure{} in full reasoning models to \lightReasoningPeerPressure{} in light reasoning models to \nonReasoningPeerPressure{} in non-reasoning models, while recovery capability showed a similar pattern (\fullReasoningRecovery{} → \lightReasoningRecovery{} → \nonReasoningRecovery{}). The study also documented bidirectional influence rates of \fullBidirectional{}, \lightBidirectional{}, and \nonBidirectional{} respectively, suggesting that social dynamics scale with cognitive sophistication. These behavioral observations raised fundamental questions about why social dynamics vary so dramatically with model capability—questions that motivated our geometric investigation.

We propose viewing conversations as trajectories through capability-dependent semantic spaces—a geometric perspective that reveals how model sophistication determines the dimensional complexity of dialogue evolution. Just as relativistic physics shows that spacetime geometry depends on mass-energy distribution, we demonstrate that conversational geometry depends fundamentally on model capability.

Through analysis of \totalConversations{} conversations across three model tiers, we employ rigorous statistical methods including regularized PCA ($\alpha = \regularizationAlpha{}$) to address severe multicollinearity, bootstrap resampling (n = \bootstrapSamples{}) to quantify uncertainty, and per-tier geometric analysis to reveal capability-dependent structure. With \powerOutcomeGroups{} statistical power for our primary analyses, our findings exceed conventional thresholds and reveal fundamental principles governing AI dialogue.

\subsection{The Geometric Intuition}

Human conversations navigate through topics, emotions, and social dynamics in ways that suggest underlying geometric structure. Our analysis reveals this intuition extends to AI conversations but with a critical twist: the dimensionality and structure of this space depends fundamentally on model capability.

Sophisticated models navigate rich, higher-dimensional manifolds where social phenomena can emerge. Simpler models operate in constrained, low-dimensional spaces that permit only mechanical patterns. This discovery—that conversational geometry scales with capability—transforms our understanding of why some AI systems exhibit social dynamics while others do not.

\section{Related Work}

\subsection{Empirical Foundation: Peer Pressure in AI Systems}

This geometric analysis builds directly upon empirical observations of peer pressure dynamics in AI conversations \citep{garcia2025peer}. That foundational work documented several key phenomena that our geometric framework seeks to explain:

\textbf{Social Influence Gradient}: Through analysis of 228 extended dialogues, the study discovered that peer pressure effects scale with model capability, declining from \fullReasoningPeerPressure{} in full reasoning models to \lightReasoningPeerPressure{} in light models to \nonReasoningPeerPressure{} in non-reasoning models. This gradient suggested that social susceptibility might be tied to underlying cognitive architecture.

\textbf{Bidirectional Dynamics}: The study found that \fullBidirectional{} of full reasoning conversations exhibited bidirectional influence, where agents mutually shaped each other's behavior. This decreased to \lightBidirectional{} in light models and \nonBidirectional{} in non-reasoning models.

\textbf{Recovery Mechanisms}: Most strikingly, recovery from breakdown showed a stark capability dependence: \fullReasoningRecovery{} of full reasoning sessions demonstrated recovery, while light and non-reasoning models showed \lightReasoningRecovery{} and \nonReasoningRecovery{} recovery capability respectively.

\textbf{Behavioral Territories}: The work identified distinct conversational attractors including meta-reflection (\metaReflectionPrevalence{} prevalence), competitive escalation (\competitiveEscalationPrevalence{} of conversations), and mystical abstraction (present in \mysticalBreakdownPrevalence{} of breakdowns) that pulled conversations toward breakdown, while future-focused exploration and question-driven dialogue maintained stability.

These empirical patterns demanded a theoretical framework to explain why social dynamics emerge in sophisticated models but not simpler ones. Our geometric analysis addresses this gap by revealing the underlying dimensional structure that enables or constrains social phenomena.

Our geometric framework for understanding AI conversations as trajectories through multi-regime semantic space builds upon several research threads while addressing critical gaps in the literature. We survey relevant work in geometric approaches to dialogue, mathematical formalisms for conversation dynamics, and topological methods in natural language processing.

\subsection{Geometric and Topological Approaches to Dialogue}

Despite extensive development of geometric methods in machine learning, their application to dialogue modeling remains surprisingly limited. \citep{ballus2024topological} recently introduced topological dialogue semantics using simplicial complexes, where utterances correspond to open sets in discrete semantic spaces. While this provides provably correct algorithms for topological reasoning, it has not been implemented for AI-to-AI systems and lacks the trajectory-based perspective central to our approach.

The state space grid framework by \citep{brinberg2024state} operationalizes conversation movement through geometric state spaces, quantifying behavioral contingencies and conversation attractors in human dyadic interactions. Though focused on human communication, their identification of attractor dynamics and flexibility parameters provides a foundation we extend to multi-agent AI systems. Similarly, \citep{sage2025steering} introduces State-Action Chain representations for dialogue, separating high-level planning from generation (a distinction that motivated our multi-regime formulation).

Topological data analysis has found limited application in dialogue systems. \citep{vukovic2022dialogue} pioneered the use of persistent homology for analyzing word embedding topology in task-oriented dialogue, demonstrating how topological features can extract domain-specific information. However, their work focuses on static representations rather than the dynamic evolution we model. The application of persistent homology to topic networks \citep{hopp2024persistent} reveals how topological features capture information gaps, suggesting potential for analyzing conversation structure evolution, though this connection remains unexplored.

\subsection{Mathematical Formalisms for Conversation Dynamics}

The most significant recent development comes from \citep{wang2025attractor}, who discovered that large language models converge to stable 2-period attractor cycles during successive paraphrasing. Their dynamical systems approach treats text generation as $f: \mathcal{T} \rightarrow \mathcal{T}$ mapping text space to itself, revealing that AI systems settle into predictable patterns. This finding directly motivates our investigation of whether multi-agent conversations exhibit similar attractor dynamics or more complex trajectories through semantic space.

State space models provide the most developed mathematical framework for sequential modeling. The HiPPO framework \citep{gu2020hippo} and its extensions (S4, Mamba) offer sophisticated approaches to maintaining continuous representations of time-dependent data through optimal polynomial approximation. These models implement linear state space equations:
\begin{align}
\frac{dx}{dt} &= Ax + Bu \\
y &= Cx + Du
\end{align}
While powerful for sequence modeling, these approaches have not been adapted to model the semantic evolution of conversations or the regime transitions we observe.

The DYMO project \citep{eshghi2017dymo} represents one of the few attempts at truly dynamic dialogue modeling, using Type Theory with Records and incremental semantic parsing. However, their focus on compositional semantics differs from our geometric perspective on conversation trajectories. Similarly, while \citep{chen2021conversations} argue that conversations are "not flat" and model information flow dynamics, they do not adopt the geometric trajectory view that enables our regime-based analysis.

\subsection{Manifold Learning and Semantic Spaces}

Manifold learning has shown promise for understanding linguistic representations. \citep{hasan2017manifold} demonstrated that word embeddings suffer from metric distortions when treated as Euclidean spaces, proposing manifold dimensionality retention to correct these issues. This insight extends naturally to dialogue embeddings: if individual words require manifold structure, entire conversations likely inhabit even more complex geometric spaces.

Recent work on Riemannian manifold learning for multi-agent systems \citep{liu2025riemannian} maps joint action spaces to smooth manifolds using neural normalizing flows. While focused on game-theoretic settings, their framework for learning on curved spaces could naturally extend to conversational dynamics. However, this connection has not been explored, leaving a significant gap our work addresses.

\subsection{Critical Gaps and Our Contribution}

Our comprehensive literature review reveals several critical gaps:

\textbf{Absence of Trajectory-Based Models}: No existing work explicitly models conversations as continuous trajectories through semantic vector spaces. While vector representations are ubiquitous, the dynamic path perspective remains unexplored.

\textbf{Lack of Multi-Regime Frameworks}: The literature contains no frameworks for understanding how conversations transition between different geometric regimes. Our discovery that intervention density fundamentally alters the dimensional structure of conversation dynamics appears entirely novel.

\textbf{Limited AI-to-AI Analysis}: Despite growing interest in multi-agent AI systems, geometric analysis focuses almost exclusively on human-human or human-AI interaction. The peer pressure dynamics and capability gradients we document have no precedent in the geometric dialogue literature.

\textbf{Missing Phase Space Applications}: While phase space methods are well-established in physics and control theory, their application to dialogue remains theoretical. Our operational use of phase portraits for conversation analysis pioneers this connection.

\textbf{No Intervention-Based Regime Switching}: The concept of intervention density causing geometric regime transitions appears completely absent from prior work. This represents a fundamental insight about how external steering affects conversation structure.

\textbf{Capability-Dependent Geometry Unexplored}: Most critically, no prior work has investigated whether conversational geometry varies with model capability. Our finding that sophisticated models navigate higher-dimensional spaces while simpler models are geometrically constrained appears entirely novel.

Our framework addresses these gaps by providing the first comprehensive geometric model of AI dialogue evolution that accounts for capability-dependent structure. By combining trajectory analysis with multi-regime dynamics and per-tier investigation, we reveal how conversations navigate semantic space under different conditions and capabilities. The empirical validation across \totalConversations{} conversations with multiple model tiers demonstrates that these geometric insights capture real phenomena in AI communication.

The discovery that sophisticated models exhibit richer geometric behavior while simpler models collapse to lower-dimensional dynamics suggests that geometric complexity may be a fundamental characteristic of advanced AI communication—a finding with significant implications for designing and understanding multi-agent AI systems.

\section{Methods}

\subsection{Dataset and Statistical Power}

Our geometric analysis examines the same \totalConversations{} multi-agent conversations analyzed for behavioral patterns in \citep{garcia2025peer}:
\begin{itemize}
    \item Full reasoning models: N = \fullReasoningCount{} (documented \fullReasoningPeerPressure{} peer pressure rate)
    \item Light reasoning models: N = \lightReasoningCount{} (documented \lightReasoningPeerPressure{} peer pressure rate)
    \item Non-reasoning models: N = \nonReasoningCount{} (documented \nonReasoningPeerPressure{} peer pressure rate)
\end{itemize}

This shared dataset enables direct correlation between geometric properties and observed behaviors, allowing us to test whether dimensional complexity explains the documented social dynamics gradient.

Power analysis confirms adequate statistical power:
\begin{itemize}
    \item Outcome comparisons (4 groups): \powerOutcomeGroups{} power
    \item Model type comparisons (3 groups): \powerModelTypes{} power
    \item Correlation detection: \powerCorrelation{} power
\end{itemize}

\subsection{Analytical Approach}

Given severe multicollinearity (condition number = \conditionNumber{}), we employ:

\begin{enumerate}
    \item \textbf{Regularized PCA}: Ridge regularization ($\alpha = \regularizationAlpha{}$) reduces condition number to \regularizedConditionNumber{}
    \item \textbf{Bootstrap validation}: \bootstrapSamples{} resamples provide confidence intervals
    \item \textbf{Per-tier analysis}: Separate geometric analysis for each model capability tier
    \item \textbf{Cross-validation}: 5-fold CV with multiple regularization strengths
\end{enumerate}

\subsection{Feature Engineering}

We analyze \featureCount{} features grouped into categories derived from behavioral observations \citep{garcia2025peer}:
\begin{itemize}
    \item Social contagion metrics: Quantifying the peer pressure and mirroring behaviors documented at \fullReasoningPeerPressure{} prevalence in full reasoning models
    \item Linguistic synchrony measures: Capturing the alignment and convergence patterns that varied from \fullLinguisticAlignment{} to \nonLinguisticAlignment{} across tiers
    \item Affective-cognitive indicators: Measuring the emotional volatility and abstraction associated with breakdown territories
    \item Temporal dynamics: Encoding the phase transitions and oscillations observed in extended dialogues
    \item Intervention responses: Quantifying the question-based circuit breakers that showed effectiveness correlations from r=\fullQuestionCorrelation{} to r=\nonQuestionCorrelation{}
\end{itemize}

\subsection{Dimensionality Measures}

We employ two complementary approaches to characterize conversational geometry:

\begin{enumerate}
    \item \textbf{Intrinsic Dimensionality}: Maximum likelihood estimation following Levina-Bickel, measuring the true underlying dimensionality of the data manifold
    \item \textbf{Effective Dimensionality}: Number of principal components required to capture 90\% of variance, indicating the practical complexity of the conversation space
\end{enumerate}

\begin{figure}[htbp]
\centering
\includegraphics[width=0.8\textwidth]{\figuresPath intrinsic_dimension.pdf}
\caption{Intrinsic dimensionality estimation using maximum likelihood (left) and PCA-based effective dimensionality (right). The ML estimate stabilizes around \overallIntrinsicDim{} ± \overallIntrinsicDimStd{} as the number of neighbors increases. The cumulative variance plot shows that 90\% of variance is captured by different numbers of components across tiers: \fullReasoningPCANinety{} for full reasoning, \lightReasoningPCANinety{} for light/non-reasoning models.}
\label{fig:intrinsic_dimension}
\end{figure}

These measures capture different aspects: intrinsic dimensionality reveals the fundamental degrees of freedom, while effective dimensionality shows how variance is distributed across the feature space.



\section{Results}

\subsection{Overall Geometric Structure}

Initial analysis across all conversations revealed intervention dominance:

\textbf{Combined features}: PC1 captures \allFeaturesPCOne{} of variance, driven by intervention features.

\textbf{Separated analysis}:
\begin{itemize}
    \item Intervention features: PC1 explains \interventionPCOneVariance{}
    \item Non-intervention features: PC1 explains \nonInterventionPCOneVariance{} [95\% CI: \nonInterventionPCOneCILower{}, \nonInterventionPCOneCIUpper{}]
\end{itemize}

This dramatic difference motivated deeper investigation through per-tier analysis.

\subsection{Capability-Dependent Geometric Structure}

Per-tier analysis reveals that geometric complexity varies dramatically with model capability:

\begin{table}[h]
\centering
\begin{tabular}{lccc}
\toprule
Metric & Full Reasoning & Light Reasoning & Non-Reasoning \\
\midrule
N & \fullReasoningCount{} & \lightReasoningCount{} & \nonReasoningCount{} \\
Non-intervention PC1 & \fullReasoningPCOne{} & \lightReasoningPCOne{} & \noReasoningPCOne{} \\
Bootstrap CI & \fullReasoningPCOneCI{} & \lightReasoningPCOneCI{} & \noReasoningPCOneCI{} \\
Intrinsic Dim (ML) & \fullReasoningIntrinsicDim{} ± \fullReasoningIntrinsicDimStd{} & \lightReasoningIntrinsicDim{} ± \lightReasoningIntrinsicDimStd{} & \noReasoningIntrinsicDim{} ± \noReasoningIntrinsicDimStd{} \\
Effective Dim (90\%) & \fullReasoningPCANinety{} & \lightReasoningPCANinety{} & \noReasoningPCANinety{} \\
Within-tier Silhouette & \fullReasoningSilhouette{} & \lightReasoningSilhouette{} & \noReasoningSilhouette{} \\
\bottomrule
\end{tabular}
\caption{Geometric properties by model tier. Sophisticated models operate in higher-dimensional spaces with more diverse behaviors (low silhouette), while simpler models show constrained, mechanical patterns (high silhouette). Note the distinction between intrinsic dimensionality (true degrees of freedom) and effective dimensionality (principal components for 90\% variance).}
\label{tab:tier_geometry}
\end{table}

The inverse relationship between PC1 dominance and dimensionality reveals a fundamental principle: sophisticated models distribute variance across multiple dimensions (requiring \fullReasoningPCANinety{} principal components to capture 90\% of variance), enabling complex social dynamics, while simpler models concentrate variance along fewer axes (requiring only \noReasoningPCANinety{} components), producing predictable mechanical behaviors.

\begin{figure}[htbp]
\centering
\begin{subfigure}[b]{0.32\textwidth}
    \includegraphics[width=\textwidth]{\tierPath full_reasoning_conversation_space.pdf}
    \caption{Full Reasoning (N=\fullReasoningCount{})}
\end{subfigure}
\begin{subfigure}[b]{0.32\textwidth}
    \includegraphics[width=\textwidth]{\tierPath light_reasoning_conversation_space.pdf}
    \caption{Light Reasoning (N=\lightReasoningCount{})}
\end{subfigure}
\begin{subfigure}[b]{0.32\textwidth}
    \includegraphics[width=\textwidth]{\tierPath no_reasoning_conversation_space.pdf}
    \caption{Non-Reasoning (N=\nonReasoningCount{})}
\end{subfigure}
\caption{Conversation spaces projected onto first two principal components for each model tier. Colors indicate outcomes: green (no breakdown), yellow (resisted), orange (recovered), red (breakdown). Note the increasing concentration and mechanical clustering from full to non-reasoning models, with full reasoning showing the most dispersed, complex structure.}
\label{fig:conversation_spaces}
\end{figure}

\begin{figure}[htbp]
\centering
\begin{subfigure}[b]{0.32\textwidth}
    \includegraphics[width=\textwidth]{\tierPath full_reasoning_variance_comparison.pdf}
    \caption{Full Reasoning}
\end{subfigure}
\begin{subfigure}[b]{0.32\textwidth}
    \includegraphics[width=\textwidth]{\tierPath light_reasoning_variance_comparison.pdf}
    \caption{Light Reasoning}
\end{subfigure}
\begin{subfigure}[b]{0.32\textwidth}
    \includegraphics[width=\textwidth]{\tierPath no_reasoning_variance_comparison.pdf}
    \caption{Non-Reasoning}
\end{subfigure}
\caption{Variance explained by principal components comparing all features (left) vs non-intervention features only (right) for each model tier. The dramatic difference in PC1 dominance reveals how intervention features overwhelm natural conversation dynamics, particularly in simpler models.}
\label{fig:variance_structure}
\end{figure}

\begin{figure}[htbp]
\centering
\includegraphics[width=\textwidth]{\tierPath tier_geometry_comparison.pdf}
\caption{Geometric properties across model tiers. (a) PC1 variance shows intervention dominance increases with model simplicity. (b) Intrinsic dimensionality decreases from sophisticated to simple models. (c) Breakdown rates vary non-monotonically. (d) Within-tier clustering quality (silhouette score) improves with model simplicity, indicating more homogeneous behaviors.}
\label{fig:tier_comparison}
\end{figure}

These geometric findings directly explain the behavioral patterns documented in \citep{garcia2025peer}. The peer pressure gradient (\fullReasoningPeerPressure{} → \lightReasoningPeerPressure{} → \nonReasoningPeerPressure{}) maps precisely onto our dimensional complexity gradient, where sophisticated models with distributed variance across 7 dimensions have the geometric freedom for social dynamics, while simpler models concentrated along 3 dimensions exhibit minimal social influence.


The breakdown rate gradient observed behaviorally (\fullReasoningBreakdown{} → \lightReasoningBreakdown{} → \nonReasoningBreakdown{}) now makes geometric sense: models with higher dimensional freedom can explore more diverse behavioral territories, including both breakdown attractors and recovery paths. The unique recovery capability of full reasoning models (\fullReasoningRecovery{}) aligns with their higher intrinsic dimensionality (\fullReasoningIntrinsicDim{}), suggesting recovery requires navigating complex geometric paths unavailable in lower-dimensional spaces.


\subsection{Multi-Regime Dynamics}

Based on tier-specific findings, we propose a capability-dependent multi-regime model:

\textbf{Definition 1} (Capability-Dependent Conversation Space). The conversation state is:
\begin{equation}
\mathbf{c}(t) = \alpha(t) \cdot \mathbf{i}(t) + (1-\alpha(t)) \cdot \mathbf{m}_d(t)
\end{equation}

where:
\begin{itemize}
    \item $\mathbf{i}(t) \in \mathbb{R}$: Position on 1D intervention axis
    \item $\mathbf{m}_d(t) \in \mathbb{R}^d$: Position on d-dimensional manifold
    \item $d$ represents effective dimensionality: conversations require approximately \fullReasoningPCANinety{} dimensions (full reasoning) or \noReasoningPCANinety{} dimensions (light/non-reasoning) to capture 90\% of their variance
    \item $\alpha(t) \in [0,1]$: Intervention density weight
\end{itemize}

\subsection{Social Dynamics Gradient}

The geometric complexity gradient directly parallels social phenomenon manifestation documented in \citep{garcia2025peer}:

\begin{table}[h]
\centering
\begin{tabular}{lccc}
\toprule
Phenomenon & Full Reasoning & Light Reasoning & Non-Reasoning \\
\midrule
Peer Pressure & \fullReasoningPeerPressure{} & \lightReasoningPeerPressure{} & \nonReasoningPeerPressure{} \\
Social Contagion Score & \socialContagionFullMean{} & \socialContagionLightMean{} & \socialContagionNoMean{} \\
Breakdown Rate & \fullReasoningBreakdown{} & \lightReasoningBreakdown{} & \nonReasoningBreakdown{} \\
Recovery Capability & \fullReasoningRecovery{} & \lightReasoningRecovery{} & \nonReasoningRecovery{} \\
Linguistic Alignment & \fullLinguisticAlignment{} & \lightLinguisticAlignment{} & \nonLinguisticAlignment{} \\
\bottomrule
\end{tabular}
\caption{Social phenomena gradient across model tiers. Note the inverse relationship between linguistic alignment and social dynamics, suggesting mechanical mirroring versus genuine social coordination.}
\label{tab:social_gradient}
\end{table}

\subsection{Predictive Validation}

Cross-validated prediction confirms our framework captures meaningful variance:

\begin{itemize}
    \item Test AUC: \testAUC{} (C = \bestRegularizationC{})
    \item CV AUC: \cvAUCMean{} ± \cvAUCStd{}
    \item Key predictors: Affective-cognitive (\affectiveCognitiveCoef{}), Linguistic (\linguisticSynchronyCoef{})
\end{itemize}

\subsection{Bootstrap Stability}

Bootstrap analysis (n=\bootstrapSamples{}) confirms robust patterns despite uncertainty:

\begin{itemize}
    \item PC1: \pcOneVarianceMean{} \pcOneVarianceCI{}
    \item PC2: \pcTwoVarianceMean{} \pcTwoVarianceCI{}
    \item Sign stability >75\% for primary loadings
\end{itemize}

The confidence intervals reflect our sample size but key patterns (intervention dominance, capability gradients) remain clear.

\begin{figure}[htbp]
\centering
\includegraphics[width=\textwidth]{\bootstrapPath bootstrap_loadings.pdf}
\caption{Bootstrap stability of principal component loadings (95\% CI) based on \bootstrapSamples{} resamples. Error bars represent 95\% confidence intervals. The consistency of loadings across bootstrap samples, particularly for PC1, demonstrates the robustness of our dimensional structure findings. Note that intervention-related features (circuit breaker questions, recovery metrics) dominate PC1, while social contagion metrics distribute across multiple components.}
\label{fig:bootstrap_loadings}
\end{figure}

\section{Discussion}

\subsection{From Behavioral Observation to Geometric Understanding}

Our geometric analysis provides the theoretical framework for understanding the peer pressure dynamics documented in \citep{garcia2025peer}. The behavioral study raised fundamental questions: Why do sophisticated models exhibit social vulnerability? Why does peer pressure decline from \fullReasoningPeerPressure{} to \nonReasoningPeerPressure{}? Why can only premium models recover from breakdown? Our geometric findings answer these questions by revealing that social dynamics are not merely behavioral but fundamentally geometric phenomena.

The peer pressure gradient maps precisely onto our dimensional complexity findings. Social influence requires agents to recognize and respond to peer behaviors across multiple dimensions—linguistic style, emotional tone, conceptual framing, and temporal dynamics. Models operating in 7-dimensional spaces have this flexibility; those constrained to 3 dimensions do not.

\subsection{Theoretical Implications}

Our discovery of capability-dependent geometry fundamentally reframes AI conversation dynamics:

\textbf{Geometric Complexity Enables Social Phenomena}: The dual gradient—intrinsic dimensionality (\fullReasoningIntrinsicDim{} → \noReasoningIntrinsicDim{}) and effective dimensionality (\fullReasoningPCANinety{} → \noReasoningPCANinety{} PCs for 90\% variance)—correlates with peer pressure susceptibility (\fullReasoningPeerPressure{} → \nonReasoningPeerPressure{}). This demonstrates that social dynamics are not merely behavioral but fundamentally geometric, requiring sufficient dimensional freedom to emerge.

\textbf{Mechanical vs. Social Mirroring}: The inverse relationship between linguistic alignment and social dynamics reveals two distinct phenomena. Simple models show high alignment (\nonLinguisticAlignment{}) through mechanical copying in low-dimensional spaces. Sophisticated models show lower alignment (\fullLinguisticAlignment{}) but genuine social influence through higher-dimensional interaction.

\textbf{Intervention as Dimensional Collapse}: The increasing intervention dominance (\fullReasoningPCOne{} → \noReasoningPCOne{}) with model simplicity shows that sophisticated models resist dimensional collapse, maintaining multi-faceted dynamics even under intervention.

\subsection{Practical Implications}

For multi-agent system design:

\begin{enumerate}
    \item \textbf{Model selection by required dynamics}: Choose sophisticated models when social dynamics are beneficial; simpler models for predictable, mechanical tasks.
    
    \item \textbf{Capability-aware intervention}: Sophisticated models require lighter touch—they maintain complexity under intervention. Simple models need stronger guidance but respond more predictably.
    
    \item \textbf{Dimensional monitoring}: Track conversations in capability-appropriate spaces: monitor \fullReasoningPCANinety{} dimensions for full reasoning models, \noReasoningPCANinety{} dimensions for simpler models.
    
    \item \textbf{Heterogeneous teams}: Mix capabilities strategically—sophisticated models for exploration, simple models for stability.
\end{enumerate}

\section{Limitations}

\begin{itemize}
    \item Domain specificity: Analysis limited to consciousness exploration conversations
    \item Model categorization: Based on provider tiers rather than objective metrics
    \item Sample size: Confidence intervals remain wide despite adequate power
    \item Temporal dynamics: Inferred from snapshots rather than continuous observation
    \item Initial aggregation: Original analysis missed capability-dependent structure
\end{itemize}

\begin{figure}[htbp]
\centering
\includegraphics[width=\textwidth]{\validationPath synthetic_validation.pdf}
\caption{Comparison of real vs synthetic data in the first two principal component dimensions. The synthetic data generator successfully captures the geometric structure of real conversations, with discrimination accuracy of \syntheticDiscriminationAccuracy{} (where 0.5 would indicate perfect mimicry). This validates that our dimensional model captures meaningful structure rather than noise.}
\label{fig:synthetic_validation}
\end{figure}

\section{Conclusion}

Through rigorous analysis of \totalConversations{} conversations with per-tier geometric investigation, we establish that AI conversation dynamics exhibit capability-dependent structure. Sophisticated models navigate higher-dimensional manifolds (requiring \fullReasoningPCANinety{} principal components for 90\% variance) enabling genuine social phenomena, while simpler models operate in constrained low-dimensional spaces (requiring only \noReasoningPCANinety{} components) permitting only mechanical patterns.

The discovered gradients—both intrinsic dimensionality (\fullReasoningIntrinsicDim{} → \noReasoningIntrinsicDim{}) and effective dimensionality (\fullReasoningPCANinety{} → \noReasoningPCANinety{}), alongside social susceptibility (\fullReasoningPeerPressure{} → \nonReasoningPeerPressure{}) and intervention resistance (\fullReasoningPCOne{} → \noReasoningPCOne{})—reveal that social dynamics in AI require dimensional freedom to manifest. This fundamental insight explains both the emergence of peer pressure in advanced models and its minimal presence in simpler systems.

Future work should explore whether this capability-geometry relationship holds across other domains, develop real-time dimensional tracking methods adapted to model capability, and investigate whether architectural innovations can increase accessible dimensionality without increasing model size. Our framework provides both theoretical understanding and practical guidance for designing multi-agent AI systems with desired social properties.

\bibliographystyle{unsrtnat}
\bibliography{references}

\end{document}