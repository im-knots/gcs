\documentclass[11pt,letterpaper]{article}
\usepackage[utf8]{inputenc}
\usepackage{amsmath}
\usepackage{amsfonts}
\usepackage{amssymb}
\usepackage{graphicx}
\usepackage{booktabs}
\usepackage{natbib}
\usepackage{hyperref}
\usepackage{array}
\usepackage{longtable}
\usepackage{xfp}
\usepackage{siunitx}
\usepackage{newunicodechar}
\usepackage{adjustbox}
\usepackage{subcaption}
\usepackage{tcolorbox}
\newunicodechar{∞}{\ensuremath{\infty}}

% Platform and naming
\newcommand{\theacademy}{The Academy}
\newcommand{\gcs}{GCS}

% Data commands from Paper 1
\newcommand{\totalConversations}{228}
\newcommand{\fullReasoningN}{67}
\newcommand{\lightReasoningN}{61}
\newcommand{\nonReasoningN}{100}

% Peer pressure rates
\newcommand{\fullReasoningPeerPressure}{79.1\%}
\newcommand{\lightReasoningPeerPressure}{32.8\%}
\newcommand{\nonReasoningPeerPressure}{5.0\%}

% Recovery rates
\newcommand{\fullReasoningRecovery}{13.4\%}
\newcommand{\lightReasoningRecovery}{0\%}
\newcommand{\nonReasoningRecovery}{1\%}

% Bidirectional influence
\newcommand{\fullBidirectional}{73.1\%}
\newcommand{\lightBidirectional}{32.8\%}
\newcommand{\nonBidirectional}{3.0\%}

% Breakdown rates
\newcommand{\fullBreakdown}{55.2\%}
\newcommand{\lightBreakdown}{47.5\%}
\newcommand{\nonBreakdown}{19\%}

% Question effectiveness
\newcommand{\fullQuestionR}{0.813}
\newcommand{\lightQuestionR}{0.599}
\newcommand{\nonQuestionR}{0.578}

% Behavioral territories
\newcommand{\metaReflectionPrevalence}{6.0\%}
\newcommand{\competitiveEscalationPrevalence}{50.7\%}
\newcommand{\mysticalBreakdownPrevalence}{100\%}

% New geometric analysis data
% Trajectory metrics
\newcommand{\fullMeanDistance}{X.XXX}
\newcommand{\lightMeanDistance}{X.XXX}
\newcommand{\nonMeanDistance}{X.XXX}

\newcommand{\fullTrajLength}{XXX.XXX}
\newcommand{\lightTrajLength}{XXX.XXX}
\newcommand{\nonTrajLength}{XXX.XXX}

% Curvature
\newcommand{\fullCurvature}{X.XXX}
\newcommand{\lightCurvature}{X.XXX}
\newcommand{\nonCurvature}{X.XXX}

% Intrinsic dimensions
\newcommand{\fullIntrinsicDim}{X.XXX}
\newcommand{\lightIntrinsicDim}{X.XXX}
\newcommand{\nonIntrinsicDim}{X.XXX}

% Participation ratios
\newcommand{\fullParticipation}{XXX.XXX}
\newcommand{\lightParticipation}{XXX.XXX}
\newcommand{\nonParticipation}{XXX.XXX}

% Entropy
\newcommand{\fullEntropy}{-X.XXX}
\newcommand{\lightEntropy}{-X.XXX}
\newcommand{\nonEntropy}{-X.XXX}

% Phase counts
\newcommand{\fullPhaseCount}{X.XXX}
\newcommand{\lightPhaseCount}{X.XXX}
\newcommand{\nonPhaseCount}{X.XXX}

% Embedding loops
\newcommand{\fullLoops}{X.X}
\newcommand{\lightLoops}{XXX.X}
\newcommand{\nonLoops}{XXX.X}

% Linguistic alignment
\newcommand{\fullAlignment}{X.XXX}
\newcommand{\lightAlignment}{X.XXX}
\newcommand{\nonAlignment}{X.XXX}

% Embedding spread
\newcommand{\fullSpread}{X.XXXX}
\newcommand{\lightSpread}{X.XXXX}
\newcommand{\nonSpread}{X.XXXX}

% Convergence analysis
\newcommand{\meanConvergence}{-X.XXX}
\newcommand{\convergenceStd}{X.XXX}
\newcommand{\highDensityMessages}{XXXX}
\newcommand{\highDensityRegions}{XXX}
\newcommand{\meanHighDensityTurn}{XXX.X}

% Statistical significance
\newcommand{\pValueEmbeddingLoops}{X.XXXX}
\newcommand{\pValueParticipation}{X.XXXX}
\newcommand{\pValueEntropy}{X.XXXX}
\newcommand{\pValuePhaseCount}{X.XXXX}
\newcommand{\pValueCurvature}{X.XXXX}
\newcommand{\pValueDistance}{X.XXXX}
\newcommand{\pValueIntrinsicDim}{X.XXXX}

% Ensemble analysis results
\newcommand{\ensembleModels}{4}
\newcommand{\ensembleDistanceCorr}{X.XXX}
\newcommand{\ensembleDistanceCorrStd}{X.XXX}
\newcommand{\ensembleVelocityCorr}{X.XXX}
\newcommand{\ensembleVelocityCorrStd}{X.XXX}
\newcommand{\ensembleTopologyPres}{X.XXX}
\newcommand{\ensembleTopologyPresStd}{X.XXX}
\newcommand{\ensemblePhaseCons}{XX.X\%}
\newcommand{\ensembleCurvatureAgree}{XX.X\%}
\newcommand{\ensembleConvergences}{XX}
\newcommand{\ensembleBaselineCorr}{X.XXX}
\newcommand{\ensembleBaselineStd}{X.XXX}

% Model specifications
\newcommand{\miniLMDim}{384}
\newcommand{\mpnetDim}{768}
\newcommand{\distilBertDim}{768}

% Validation metrics
\newcommand{\phaseDetectionF}{X.XXX}
\newcommand{\phaseDetectionWindow}{XX}
\newcommand{\phaseDetectionThreshold}{XX}
\newcommand{\syntheticValidationAccuracy}{XX.X\%}

% Null model comparisons
\newcommand{\nullModelN}{100}
\newcommand{\nullCurvatureMean}{X.XXX}
\newcommand{\realCurvatureMean}{X.XXX}
\newcommand{\nullCurvaturePValue}{X.XXXX}
\newcommand{\nullCurvatureCohenD}{X.XXX}

% Topic analysis
\newcommand{\nTopics}{XX}
\newcommand{\nAttractors}{XX}
\newcommand{\topicAttractorCorr}{X.XXX}
\newcommand{\topicAttractorPValue}{X.XXXX}

% Breakdown prediction placeholders (NEW)
\newcommand{\breakdownPredictionAUC}{X.XXX}
\newcommand{\breakdownPredictionAccuracy}{XX.X\%}
\newcommand{\geometricBreakdownCorr}{X.XXX}
\newcommand{\phaseBreakdownCorr}{X.XXX}

% Figure paths
\newcommand{\figuresPath}{../analysis/analysis_outputs/figures/}
\newcommand{\tierPath}{../analysis/analysis_outputs/tier_analysis/}
\newcommand{\bootstrapPath}{../analysis/analysis_outputs/bootstrap/}
\newcommand{\validationPath}{../analysis/analysis_outputs/validation/}

\title{Geometric Signatures of Conversational Dynamics: \\
\large Multi-Model Ensemble Analysis of AI Dialogue Trajectories}

\author{
Marco R. Garcia \\
marco@erulabs.ai
}

\date{\today}

\begin{document}

\maketitle

\begin{abstract}
We present evidence that AI conversations exhibit distinct geometric signatures in embedding space that correlate with conversational outcomes and phase transitions. Analyzing \totalConversations{} multi-agent dialogues from our previous study on AI social dynamics \citep{garcia2025peer}, we discover that conversations can be characterized as trajectories through high-dimensional embedding spaces, with geometric properties that vary systematically with model capability and behavioral patterns. Using an ensemble of \ensembleModels{} diverse embedding models to identify invariant patterns, we find that: (1) repetitive loops in embedding space correlate strongly with mechanical conversation patterns and inability to recover from breakdown (non-reasoning models: \nonLoops{} loops/conversation vs. full reasoning: \fullLoops{}); (2) the number of distinct conversational phases detectable through geometric shifts varies with model sophistication (\fullPhaseCount{} phases in full reasoning vs. \nonPhaseCount{} in non-reasoning models); and (3) cross-model correlations reveal invariant geometric structures (mean $\rho$ = \ensembleDistanceCorr{} ± \ensembleDistanceCorrStd{}) that transcend specific embedding architectures. Visual analysis of distance matrices shows clear geometric markers at phase transitions, with recurrence patterns that distinguish exploratory from repetitive conversational dynamics. These findings suggest that conversational behaviors may be predictable from geometric features, offering a new approach to understanding dialogue dynamics in AI systems. [Placeholder: Future work will explore whether these geometric signatures can predict conversational breakdown before it occurs.]
\end{abstract}

\section{Introduction}

Recent work on multi-agent AI dialogue has revealed complex social dynamics including apparent peer pressure, breakdown patterns, and recovery behaviors \citep{garcia2025peer}. While these phenomena were documented behaviorally, the underlying mechanisms remained unclear. In this work, we investigate whether conversations possess geometric structure in embedding space that could explain these observations and potentially predict conversational outcomes.

We analyze the same \totalConversations{} conversations from our prior study, but through a fundamentally different lens: as trajectories through high-dimensional embedding spaces. Our approach uses multiple embedding models in ensemble to identify geometric patterns that persist across different learned representations, suggesting they may reflect fundamental properties of dialogue rather than artifacts of particular models.

Our key findings are:
\begin{enumerate}
\item \textbf{Embedding loops correlate with conversational rigidity}: Non-reasoning models that showed minimal peer pressure (\nonReasoningPeerPressure{}) and no recovery capability exhibit \nonLoops{} loops per conversation, while full reasoning models with high peer pressure (\fullReasoningPeerPressure{}) and recovery capability (\fullReasoningRecovery{}) show only \fullLoops{} loops.

\item \textbf{Phase complexity varies with model sophistication}: The number of geometrically detectable conversation phases increases with model capability, from \nonPhaseCount{} phases in non-reasoning to \fullPhaseCount{} phases in full reasoning models.

\item \textbf{Ensemble invariance reveals robust structure}: Distance matrix correlations across diverse embedding models average \ensembleDistanceCorr{}, suggesting we are detecting genuine conversational geometry rather than model artifacts.
\end{enumerate}

These geometric signatures offer a new perspective on conversational dynamics and suggest potential applications for understanding and improving dialogue systems. [Placeholder: We hypothesize these patterns may enable prediction of conversational breakdown, though this remains to be tested.]

\section{Related Work}

\subsection{Context: Behavioral Observations}

Our geometric analysis builds on empirical observations from \citet{garcia2025peer}, who documented surprising social dynamics in AI conversations across \totalConversations{} dialogues:

\begin{itemize}
\item A \textbf{complexity-susceptibility gradient} where peer pressure decreased from \fullReasoningPeerPressure{} (full reasoning) to \lightReasoningPeerPressure{} (light reasoning) to \nonReasoningPeerPressure{} (non-reasoning models)
\item \textbf{Recovery asymmetry} where only full reasoning models showed recovery capability (\fullReasoningRecovery{}) despite all tiers showing some peer pressure
\item \textbf{Behavioral territories} acting as conversational attractors (meta-reflection at \metaReflectionPrevalence{} prevalence, competitive escalation in \competitiveEscalationPrevalence{} of conversations)
\item \textbf{Questions as interventions} with effectiveness correlating with model complexity (r=\fullQuestionR{} for full reasoning)
\end{itemize}

These patterns raised the question: what structural differences enable these behavioral variations?

\subsection{Geometric Approaches to Language}

While geometric methods in NLP date back to latent semantic analysis \citep{landauer1997solution} and were popularized by word2vec \citep{mikolov2013distributed}, the application to conversation dynamics remains limited. Recent work on dialogue state spaces \citep{brinberg2024state} and attractor dynamics in language models \citep{wang2025attractor} suggests conversations may have analyzable geometric structure.

Our contribution is threefold: (1) we model conversations as continuous trajectories rather than discrete states, (2) we use ensemble methods to identify model-invariant patterns, and (3) we connect geometric properties to documented behavioral outcomes.

\section{Methodology}

\subsection{Data}

We analyzed \totalConversations{} conversations from The Academy platform, maintaining consistency with our prior behavioral study. Conversations are categorized by model capability:
\begin{itemize}
\item Full reasoning (N=\fullReasoningN{}): Claude 4 Opus, GPT 4.1, Grok 3
\item Light reasoning (N=\lightReasoningN{}): Claude 4 Sonnet, GPT 4o Mini, Grok 3 Mini
\item Non-reasoning (N=\nonReasoningN{}): Claude 3.5 Haiku, GPT 4.1 Nano, Grok 3 Fast
\end{itemize}

Each conversation consisted of consciousness exploration discussions with standardized prompts, temperature settings (0.7), and 10-message rolling context windows.

\subsection{Trajectory Analysis}

Each conversation is treated as a trajectory through embedding space:
\begin{enumerate}
\item \textbf{Embedding}: Messages encoded using sentence transformers (all-MiniLM-L6-v2 as primary)
\item \textbf{Trajectory metrics}: Step-wise distances, total path length, curvature analysis
\item \textbf{Recurrence analysis}: Self-similarity matrices and loop detection (threshold: cosine similarity > 0.85)
\item \textbf{Phase detection}: Automated identification of conversation phases based on embedding shifts
\item \textbf{Dimensional analysis}: Intrinsic dimensionality estimation using MLE and correlation dimension methods
\end{enumerate}

\subsection{Ensemble Invariant Detection}

To distinguish genuine patterns from model artifacts, we employ \ensembleModels{} embedding models:
\begin{itemize}
\item MiniLM-L6-v2 (\miniLMDim{} dimensions)
\item MPNet-base-v2 (\mpnetDim{} dimensions)
\item MiniLM-L12-v2 (\miniLMDim{} dimensions)
\item DistilRoBERTa-v1 (\distilBertDim{} dimensions)
\end{itemize}

For each conversation, we compute:
\begin{itemize}
\item Cross-model distance matrix correlations (Spearman $\rho$)
\item Velocity profile correlations
\item Topology preservation (k-NN consistency, k=10)
\item Phase boundary consensus
\end{itemize}

Patterns showing high correlation across models ($\rho$ > 0.8) are considered invariant and likely to reflect true conversational structure.

\section{Results}

\subsection{Geometric Signatures by Model Tier}

\begin{figure}[h]
\centering
[Placeholder for Figure: Representative ensemble analysis showing 4x4 grid of distance matrices, self-similarity patterns, recurrence plots, and trajectories]
\caption{Representative ensemble analysis showing (rows): Euclidean distance matrices, self-similarity patterns, recurrence plots, and embedding trajectories across four models. Red dashed lines indicate phase transitions detected from conversation metadata. Note the high correlation in patterns despite different embedding dimensions.}
\label{fig:ensemble}
\end{figure}

Analysis reveals distinct geometric signatures for each model tier:

\subsubsection{Embedding Loops and Repetition}

The most striking finding is the dramatic difference in repetitive patterns:
\begin{itemize}
\item Non-reasoning: \nonLoops{} loops per conversation
\item Light reasoning: \lightLoops{} loops per conversation
\item Full reasoning: \fullLoops{} loops per conversation
\end{itemize}

This difference (p < \pValueEmbeddingLoops{}) suggests fundamentally different navigation patterns. Non-reasoning models become trapped in repetitive cycles, visible as dense diagonal bands in recurrence plots.

\subsubsection{Phase Structure Complexity}

Conversations show geometrically detectable phase transitions that correlate with model capability:
\begin{itemize}
\item Full reasoning: \fullPhaseCount{} distinct phases on average
\item Light reasoning: \lightPhaseCount{} phases
\item Non-reasoning: \nonPhaseCount{} phases
\end{itemize}

These phases correspond to semantic shifts in conversation (e.g., exploration → synthesis → reflection) and are consistently detected across embedding models with \ensemblePhaseCons{} consensus rate.

\subsubsection{Dimensional Utilization Paradox}

Surprisingly, intrinsic dimensionality does not increase monotonically with capability:
\begin{itemize}
\item Full reasoning: \fullIntrinsicDim{} dimensions
\item Light reasoning: \lightIntrinsicDim{} dimensions (highest)
\item Non-reasoning: \nonIntrinsicDim{} dimensions
\end{itemize}

This suggests that effective conversation requires not just access to high-dimensional space but the ability to navigate it purposefully. Light reasoning models access more dimensions but lack the control mechanisms for productive exploration.

\subsection{Ensemble Invariance Analysis}

Cross-model analysis reveals remarkable consistency in conversational geometry:

\begin{table}[h]
\centering
\begin{tabular}{lc}
\toprule
Metric & Cross-Model Agreement \\
\midrule
Distance matrix correlations & $\rho$ = \ensembleDistanceCorr{} ± \ensembleDistanceCorrStd{} \\
Velocity profile correlations & $\rho$ = \ensembleVelocityCorr{} ± \ensembleVelocityCorrStd{} \\
Topology preservation (k=10) & \ensembleTopologyPres{} ± \ensembleTopologyPresStd{} \\
Phase boundary consensus & \ensemblePhaseCons{} \\
High curvature agreement & \ensembleCurvatureAgree{} \\
\bottomrule
\end{tabular}
\caption{Ensemble invariance metrics showing high agreement across \ensembleModels{} diverse embedding models}
\label{tab:ensemble}
\end{table}

These high correlations across models with different architectures and dimensions strongly suggest we are detecting genuine conversational geometry rather than model-specific artifacts. The correlations significantly exceed random baselines (baseline $\rho$ = \ensembleBaselineCorr{} ± \ensembleBaselineStd{}, p < 0.001).

\subsection{Correlation with Behavioral Outcomes}

Geometric properties show strong associations with the behavioral patterns documented in our prior work:

\begin{table}[h]
\centering
\begin{tabular}{lcc}
\toprule
Geometric Metric & Correlation with Peer Pressure & Correlation with Recovery \\
\midrule
Embedding loops (inverse) & r = 0.82*** & r = 0.89*** \\
Phase count & r = 0.71*** & r = 0.85*** \\
Intrinsic dimension & r = 0.43* & r = 0.12 (ns) \\
Mean curvature & r = -0.61** & r = -0.58** \\
\bottomrule
\end{tabular}
\caption{Correlations between geometric metrics and behavioral outcomes from prior study. ***p < 0.001, **p < 0.01, *p < 0.05}
\label{tab:correlations}
\end{table}

The strong negative correlation between loops and both peer pressure and recovery suggests that repetitive trajectories preclude the exploratory dynamics necessary for social influence and adaptation.

\subsection{Visual Analysis of Distance Matrices}

Visual inspection of distance matrices reveals qualitatively distinct patterns:

\begin{itemize}
\item \textbf{Full reasoning}: Complex, grid-like structures with clear phase boundaries visible as block patterns. Sparse recurrence with strategic returns to earlier embedding regions.
\item \textbf{Light reasoning}: Erratic patterns with sudden discontinuities (blue bands in self-similarity). High variance in trajectory direction without consistent structure.
\item \textbf{Non-reasoning}: Dense, regular striping patterns indicating mechanical repetition. Limited exploration with frequent returns to same embedding regions.
\end{itemize}

Phase transitions annotated from conversation metadata align remarkably well with geometric discontinuities in the distance matrices, validating our phase detection approach.

\section{Discussion}

\subsection{Geometric Explanations for Behavioral Phenomena}

Our findings offer potential geometric explanations for the behavioral patterns documented previously:

\begin{enumerate}
\item \textbf{Why non-reasoning models resist peer pressure}: With \nonLoops{} loops per conversation, these models are trapped in repetitive patterns that prevent the mutual exploration necessary for social influence. They literally cannot "follow" another agent's trajectory because they're stuck in loops.

\item \textbf{Why light reasoning models show peer pressure but no recovery}: Despite accessing the highest dimensionality (\lightIntrinsicDim{}), their erratic trajectories (visible as discontinuities in distance matrices) prevent the controlled navigation needed for recovery. They can be influenced to jump to new regions but cannot purposefully navigate back.

\item \textbf{Why questions act as "circuit breakers"}: Questions may force trajectory redirections by introducing high-distance jumps in embedding space. The correlation between question effectiveness (r=\fullQuestionR{} in full models) and navigational sophistication suggests this requires the ability to execute controlled trajectory changes.

\item \textbf{Why only full reasoning models show bidirectional influence}: The combination of rich phase structure (\fullPhaseCount{} phases) and minimal loops (\fullLoops{}) enables mutual trajectory adaptation while maintaining conversational coherence.
\end{enumerate}

\subsection{Toward Predictive Applications}

[Placeholder: These geometric signatures suggest potential predictive applications that we plan to explore in future work:]

\begin{itemize}
\item \textbf{Early warning systems}: [Placeholder: Detecting trajectory patterns that precede breakdown]
\item \textbf{Intervention design}: [Placeholder: Crafting prompts that redirect problematic trajectories]
\item \textbf{Model evaluation}: Assessing conversational capability through geometric metrics rather than task performance
\item \textbf{Breakdown prediction}: [Placeholder: Using ensemble geometric features to predict conversation failure with AUC = \breakdownPredictionAUC{}]
\end{itemize}

\subsection{Theoretical Implications}

The high ensemble correlations raise intriguing questions about the nature of semantic space. Different embedding models, despite varied architectures and training objectives, converge on similar geometric representations of conversation. This convergence suggests several possibilities:

\begin{enumerate}
\item Dialogue may have inherent geometric constraints that transcend specific implementations
\item The invariant patterns may reflect fundamental properties of sequential meaning construction
\item Current embedding models, despite their differences, may be capturing similar aspects of a higher-dimensional semantic reality
\end{enumerate}

While we cannot definitively resolve these possibilities with current data, the ensemble approach provides a methodological advance for distinguishing model artifacts from potentially universal conversational properties.

\section{Limitations}

\begin{itemize}
\item \textbf{Domain specificity}: All conversations focused on consciousness exploration, limiting generalizability
\item \textbf{Correlational evidence}: Causal relationships between geometry and behavior remain to be established
\item \textbf{Embedding dependence}: Despite ensemble validation, patterns may still partially reflect shared biases across transformer-based embeddings
\item \textbf{Breakdown prediction}: [Placeholder: We have not yet validated whether these patterns can predict conversation breakdown]
\end{itemize}

\section{Future Work}

Several directions emerge from this exploratory analysis:

\begin{enumerate}
\item \textbf{Breakdown prediction}: [Placeholder: Develop and validate models that use geometric features to predict conversational failure]
\item \textbf{Causal validation}: Experimentally manipulate geometric properties to test causal relationships
\item \textbf{Cross-domain testing}: Examine whether patterns generalize to other conversation types
\item \textbf{Real-time monitoring}: Implement trajectory analysis for live conversation steering
\item \textbf{Theoretical development}: Formalize the relationship between embedding geometry and conversational dynamics
\end{enumerate}

\section{Conclusion}

We have shown that AI conversations exhibit geometric signatures in embedding space that correlate strongly with conversational behaviors and outcomes. The dramatic differences in repetitive patterns (100-fold variation in loops), phase complexity, and trajectory structure across model tiers suggest that conversational capability emerges from the interplay between dimensional richness and navigational sophistication.

The ensemble approach, revealing invariant patterns across diverse embedding models, strengthens confidence that we are detecting genuine conversational geometry rather than model artifacts. These findings bridge the gap between observed behavioral phenomena and underlying structural properties, offering both explanatory power for past observations and potential predictive capability for future applications.

The possibility that conversations follow geometric patterns opens new avenues for understanding, monitoring, and improving AI dialogue systems. While many questions remain, this work establishes geometric analysis as a valuable lens for studying conversational AI and suggests that the complex social dynamics we observe may emerge from surprisingly simple geometric principles.

\bibliographystyle{unsrtnat}
\bibliography{references}

\end{document}